\section{Relativistic Quantum Mechanics}\label{sec:chapter2}

\subsection{Quantum Mechanics}\label{susec:2_1}

\subsection{Symmetries}\label{susec:2_2}
\subsubsection{\enquote{For this to be unitary and linear, $t$ must be Hermitian and linear} \page{51}}\label{sususec:2_2_p51_1}
Linearity is trivial and hermiticity follow from the following observation:
\begin{align*} 
	\innerproduct{U\Psi}{U\Phi}=&\innerproduct{(1+i\varepsilon t) \Psi}{(1+i\varepsilon t) \Phi}\\
	=&\innerproduct{\Psi}{\Phi}+\varepsilon i\left(\innerproduct{\Psi}{t \Phi}-\innerproduct{t\Psi}{ \Phi}\right)+\order{\varepsilon^2}\\
	\overset{\eq{2.2.2}}{\Leftrightarrow}&\innerproduct{\Psi}{t \Phi}=\innerproduct{t\Psi}{\Phi}\\
	\overset{\eq{2.1.5}}{\Leftrightarrow}&t^\dagger=t
\end{align*}

\subsubsection{\eq{2.2.19} \page{54}}
$f^a_{bc}$ and $f^a$ have to be real as $\theta^a$ are real.

\subsubsection{\eq{2.2.21} \page{54}}
From \eq{2.2.20} we have up to $\order{\theta^2,\bar{\theta}^2}$
\begin{align*} 
	&1+i\left(\theta^a+\bar{\theta}^a+f^a_{\hphantom{a} bc}\bar{\theta}^b\theta^c\right)t_a+
	\dfrac{1}{2}\left(\theta^b+\bar{\theta}^b\right)\left(\theta^c+\bar{\theta}^c\right)t_{bc}\\
	&=\left[1+i\bar{\theta}^a t_a+\dfrac{1}{2}\bar{\theta}^b\bar{\theta}^c t_{bc}\right]\cdot\left[1+i{\theta}^a t_a+\dfrac{1}{2}{\theta}^b{\theta}^c t_{bc}\right]\\
	&=1+i{\theta}^a t_a+\dfrac{1}{2}{\theta}^b{\theta}^c t_{bc}+i\bar{\theta}^a t_a-\bar{\theta}^b t_b \theta^c t_c+\dfrac{1}{2}\bar{\theta}^b\bar{\theta}^c t_{bc}\\
	\Leftrightarrow &i f^a_{\hphantom{a} bc}\bar{\theta}^b\theta^c t_a + \dfrac{1}{2}\left(\theta^b \bar{\theta}^c+ \bar{\theta}^b\theta^c\right)t_{bc}=-\bar{\theta}^b  \theta^c t_b t_c\\
	\Leftrightarrow&\bar{\theta}^b\theta^c\left[t_{bc}+ i f^a_{\hphantom{a} bc} t_a+t_b t_c\right]=0\\
	\Leftrightarrow& t_{bc}=- i f^a_{\hphantom{a} bc} t_a-t_b t_c
\end{align*}

\subsubsection{\eq{2.2.22} \page{54}}
\begin{align*} 
	- i f^a_{\hphantom{a} bc} t_a-t_b t_c \overset{\eqc{2.2.21}}{=}t_{bc}&=t_{cb} \overset{\eqc{2.2.21}}{=} - i f^a_{\hphantom{a} cb} t_a-t_c t_b\\
	\Leftrightarrow\commutator{t_b}{t_c}&=i\left(f^a_{\hphantom{a} cb}-f^a_{\hphantom{a} bc}\right)t_a
\end{align*}


\subsection{Quantum Lorentz Transformations}\label{susec:2_3}
\subsubsection{\enquote{$\Lambda^\mu_{\hphantom{\mu}\nu}$ has an \textbf{inverse}} \page{57}}
This is true because \[\det(\Lambda)=\pm 1\neq 0\]

\subsubsection{\enquote{$\left(\bar{\Lambda}\Lambda\right)^0_{\hphantom{0}0}\geq \bar{\Lambda}^0_{\hphantom{0}0}\Lambda^0_{\hphantom{0}0}-\sqrt{\left(\Lambda^0_{\hphantom{0}0}\right)^2-1}\sqrt{\left(\bar{\Lambda}^0_{\hphantom{0}0}\right)^2-1}\geq1$} \page{58}}
The first inequality is trivial from the preceding considerations. For the second inequality assume the contrary holds, then:
\begin{align*} 
	0&\leq\bar{\Lambda}^0_{\hphantom{0}0}\Lambda^0_{\hphantom{0}0}-1<\sqrt{\left(\Lambda^0_{\hphantom{0}0}\right)^2-1}\sqrt{\left(\bar{\Lambda}^0_{\hphantom{0}0}\right)^2-1}\\
	\Rightarrow &\left(\bar{\Lambda}^0_{\hphantom{0}0}\right)^2 \left(\Lambda^0_{\hphantom{0}0}\right)^2 -2 \bar{\Lambda}^0_{\hphantom{0}0}\Lambda^0_{\hphantom{0}0}+1\\
	&<\left(\bar{\Lambda}^0_{\hphantom{0}0}\right)^2 \left(\Lambda^0_{\hphantom{0}0}\right)^2-\left(\bar{\Lambda}^0_{\hphantom{0}0}\right)^2-\left(\Lambda^0_{\hphantom{0}0}\right)^2+1\\
	\Rightarrow&\left(\Lambda^0_{\hphantom{0}0}+\bar{\Lambda}^0_{\hphantom{0}0}\right)^2=\left(\bar{\Lambda}^0_{\hphantom{0}0}\right)^2-2\bar{\Lambda}^0_{\hphantom{0}0}\Lambda^0_{\hphantom{0}0} +\left(\Lambda^0_{\hphantom{0}0}\right)^2<0
\end{align*}
Which is a contradiction as $\Lambda^0_{\hphantom{0}0}+\bar{\Lambda}^0_{\hphantom{0}0}\geq 1+1=2$ and therefore completes the proof.


\subsection{The Poincaré Algebra}\label{susec:2_4}
\subsubsection{\enquote{In order fo $U(1+\omega,\varepsilon)$ to be unitary, the operators $J^{\rho\sigma}$ and $P^\rho$ must be \textbf{Hermitian}} \page{59}}
Analog to \ref{sususec:2_2_p51_1}.

\subsubsection{\eqs{2.4.8/9} \page{60}}\label{sususec:2_4_p60_1}
\begin{align*} 
	&\dfrac{1}{2}\omega_{\rho\sigma} U J^{\rho\sigma} U^{-1}-\varepsilon_\rho U P^\rho U^{-1}\\
	&\overset{\eqc{2.4.7}}{=}\dfrac{1}{2}\left(\Lambda\omega\Lambda^{-1}\right)_{\mu\nu}J^{\mu\nu}-\left(\Lambda \varepsilon-\Lambda\omega\Lambda^{-1}a\right)_\mu P^\mu\\
	&=\dfrac{1}{2}\Lambda_\mu^{\hphantom{\mu}\rho}\omega_{\rho \sigma}\left(\Lambda^{-1}\right)^\sigma_{\hphantom{\sigma}\nu}J^{\mu\nu}\\
	&-\left(\Lambda_\mu^{\hphantom{\mu}\rho} \varepsilon_\rho-\Lambda_\mu^{\hphantom{\mu}\rho}\omega_{\rho \sigma}\left(\Lambda^{-1}\right)^\sigma_{\hphantom{\sigma}\nu}a^\nu\right) P^\mu\\
	&\overset{\eqc{2.3.10}}{=}\dfrac{1}{2}\Lambda_\mu^{\hphantom{\mu}\rho}\omega_{\rho \sigma}\Lambda^{\hphantom{\nu}\sigma}_{\nu}J^{\mu\nu}\\
	&-\left(\Lambda_\mu^{\hphantom{\mu}\rho} \varepsilon_\rho-\Lambda_\mu^{\hphantom{\mu}\rho}\omega_{\rho \sigma}\Lambda^{\hphantom{\nu}\sigma}_{\nu}a^\nu\right) P^\mu\\
	&=\dfrac{1}{2}\omega_{\rho \sigma}\left(\Lambda_\mu^{\hphantom{\mu}\rho}\Lambda^{\hphantom{\nu}\sigma}_{\nu}J^{\mu\nu}+\Lambda_\mu^{\hphantom{\mu}\rho}\Lambda^{\hphantom{\nu}\sigma}_{\nu}a^\nu P^\mu\right)\\
	&-\varepsilon_\rho \Lambda_\mu^{\hphantom{\mu}\rho}P^\mu
\end{align*}
In order to be able to compare coefficients in this, the coefficient of $\omega_{\rho \sigma}$ has to be anti symmatrized. With this \eqs{2.4.8/9} follow immediately.

\subsubsection{\eqs{2.4.10/11} \page{60}}
Up to $\order{\omega,\varepsilon}$ one can identify \[U^{-1}(1+\omega,\varepsilon)=U(1-\omega,-\varepsilon)\],since \[U(1+\omega,\varepsilon)U(1-\omega,-\varepsilon)=U(1-\omega+\omega,-\varepsilon+\varepsilon)=U(1,0).\]
With this we have up to $\order{\omega,\varepsilon}$
\begin{align*} 
	&i\commutator{\dfrac{1}{2}\omega_{\mu \nu}J^{\mu\nu}-\varepsilon_\mu P^\mu}{J^{\rho\sigma}}\\
	&=\left(1+\dfrac{1}{2}i\omega_{\mu \nu}J^{\mu\nu}-i\varepsilon_\mu P^\mu\right)J^{\rho\sigma}\\
	&\cdot\left(1-\dfrac{1}{2}i\omega_{\mu \nu}J^{\mu\nu}+i\varepsilon_\mu P^\mu\right)-J^{\rho\sigma}\\
	&=U J^{\rho\sigma}U^{-1}-J^{\rho\sigma}\\
	&\overset{\eqc{2.4.8}}{=}(1+\omega)_\mu^{\hphantom{\mu}\rho}(1+\omega)_\nu^{\hphantom{\nu}\sigma}\\
	&\cdot\left(J^{\mu \nu}-\varepsilon^\mu P^\nu+\varepsilon^\nu P^\mu\right)-J^{\rho\sigma}\\
	&=-\varepsilon^\rho P^\sigma+\varepsilon^\sigma P^\rho+\omega_\nu^{\hphantom{\nu}\sigma}J^{\rho \nu}+\omega_\mu^{\hphantom{\mu}\rho}J^{\mu \sigma}
\end{align*}
and also
\begin{align*} 
	&i\commutator{\dfrac{1}{2}\omega_{\mu \nu}J^{\mu\nu}-\varepsilon_\mu P^\mu}{P^{\rho}}\\
	&=\left(1+\dfrac{1}{2}i\omega_{\mu \nu}J^{\mu\nu}-i\varepsilon_\mu P^\mu\right)P^{\rho}\\
	&\cdot\left(1-\dfrac{1}{2}i\omega_{\mu \nu}J^{\mu\nu}+i\varepsilon_\mu P^\mu\right)-P^{\rho}\\
	&=U P^{\rho}U^{-1}-P^{\rho}\\
	&\overset{\eqc{2.4.9}}{=}\omega_\mu^{\hphantom{\mu}\rho}P^\mu
\end{align*}
\subsubsection{\eqs{2.4.12/13/14} \page{60}}
The only difficulty lies in the derivation of \eq{2.4.12} because for this one has to anti symmatrize the coefficient of $\omega_{\mu \nu}$ in \eq{2.4.10} in order to be able to compare coefficients, similar to \ref{sususec:2_4_p60_1}. (Similarly if one derives \eq{2.4.13} from \eq{2.4.11})

\subsubsection{\eqs{2.4.18-24} \page{61}}
First observe \[J^{ij}=\varepsilon_{ijk}J_k\]
and \[J_i=\dfrac{1}{2}\varepsilon_{lmi}J^{lm}\]
\begin{align*} 
	\commutator{J_i}{J_j}&=\dfrac{\varepsilon_{lmi}\varepsilon_{kpj}}{4}\commutator{J^{lm}}{J^{kp}}\\
	&\overset{\eqc{2.4.12}}{=}-i\dfrac{\varepsilon_{lmi}\varepsilon_{kpj}}{4}\left[\eta^{mk}J^{lp}-\eta^{lk}J^{mp}-\eta^{pl}J^{km}+\eta^{pm}J^{kl}\right]\\
	&=-\dfrac{i}{2}\left[\varepsilon_{kil}\varepsilon_{kmj}J^{lm}+\varepsilon_{kim}\varepsilon_{kjl}J^{lm}\right]\\
	&=-\dfrac{i}{2}\left[J^{ji}-J^{ij}\right]\\
	&=iJ^{ij}=i\varepsilon_{ijk}J_k\\
	\commutator{J_i}{K_j}&=\commutator{J^{lm}}{J^{0j}}\dfrac{\varepsilon_{lmi}}{2}\\
	&\overset{\eqc{2.4.12}}{=}-i\dfrac{\varepsilon_{lmi}}{2}\left[\eta^{m0}J^{lj}-\eta^{l0}J^{mj}-\eta^{jl}J^{0m}+\eta^{jm}J^{0l}\right]\\
	&=-i\dfrac{\varepsilon_{lmi}}{2}\left[\delta_{jm} K_l-\delta_{jl}K_m\right]=i\varepsilon_{ijl}K_l\\
	\commutator{K_i}{K_j}&=\commutator{J^{0i}}{J^{0j}}\\
	&\overset{\eqc{2.4.12}}{=}-i\left[\eta^{i0}J^{0j}-\eta^{00}J^{ij}-\eta^{j0}J^{0i}+\eta^{ij}J^{00}\right]\\
	&=-iJ^{ij}=-i\varepsilon_{ijk}J_k\\
	\comm{J_i}{P_j}&=\dfrac{\varepsilon_{lmi}}{2}\comm{J^{lm}}{P^{j}}\\
	&\overset{\eqc{2.4.13}}{=}\dfrac{i\varepsilon_{lmi}}{2}\left[\eta^{jl}P^m-\eta^{jm}P^l\right]\\
	&=\dfrac{i}{2}\left[\varepsilon_{jmi}P^m-\varepsilon_{mji}P^m\right]=i\varepsilon_{ijm}P_m\\
	\commutator{K_i}{P_j}&=\commutator{J^{0i}}{P^{j}}\\
	&\overset{\eqc{2.4.13}}{=}i\left[\eta^{j0}P^i-\eta^{ji}P^0\right]\\
	&=-i\delta_{ji}P^0=-i\delta_{ij}H\\
	\comm{J_i}{H}&\overset{\eqc{2.4.13}}{=}\dfrac{i\varepsilon_{lmi}}{2}\left[\eta^{0l}P^m-\eta^{0m}P^l\right]=0\\
	\comm{P_i}{H}&=\commutator{P^i}{P^0}\overset{\eqc{2.4.14}}{=}0\\
	\comm{H}{H}&=\commutator{P^0}{P^0}\overset{\eqc{2.4.14}}{=}0	\\
	\commutator{K_i}{H}&=\commutator{J^{0i}}{P^{0}}\\
	&\overset{\eqc{2.4.13}}{=}i\left[\eta^{00}P^i-\eta^{0i}P^0\right]=-i P_i
\end{align*}

\subsubsection{\eq{2.4.27} \page{61}}
When trying to check this for e.g. the standard representation in 4 dimensions, in terms of infinitesimal rotations, attention with the sign in front of $\sin$ for the different rotation axis (see remark at the end of\ref{sususec:2_5_p73_1}).

\subsubsection{\enquote{Inspection of \eqs{2.4.18-24} shows that these commutation relations have a limit for $v\ll1$ of the form \dots} \page{62}}
Always equate same orders in $v$ for this.

\subsubsection{\enquote{$\exp(-i \tvec{K}\cdot\tvec{v})\exp(-i \tvec{P}\cdot\tvec{a})=\exp(i M\tvec{a}\cdot\tvec{v}/2)\exp(-i \left(\tvec{K}\cdot\tvec{v}+\tvec{P}\cdot\tvec{a}\right))$} \page{62}}
Use BCH Formula
\begin{align*} 
	&\exp(-i K_i v_i)\exp(-i P_j a_j)\\
	&=\exp(-i(K_i v_i+P_j a_j)+\dfrac{1}{2}(-i)^2\commutator{K_i}{P_j}v_ia_j+0)\\
	&=\exp(-i(K_i v_i+P_j a_j)+\dfrac{1}{2}iM v_i a_i)
\end{align*}

\subsection{One-Particle States}\label{susec:2_5}
\subsubsection{\eq{2.5.2} \page{63}}
$\Lambda^{-1}$ shows up here since in \eq{2.4.9} $U P U^{-1}$ is given but here $U^{-1} P U$ is being used.

\subsubsection{\enquote{with $\sigma$ within any one block by themselves furnish a representation of the inhomogeneous Lorentz group} \page{63}}
In this case the Blocks do not mix with other blocks.

\subsubsection{\enquote{and for $p^2\leq0$, also the sign of $p^0$} \page{64}}
For $p^2\leq 0$ we have
\begin{align*} 
	p^2&=-\left(p^0\right)^2+\vec{p}^2\leq0\\
	\Rightarrow \abs{\vec{p}}&\leq \abs{p^0}
\end{align*}
and from \eq{2.3.13} we know
\[\abs{\Lambda^0_{\hphantom{0}0}}\geq \abs{\vec{\Lambda^{0}_{\hphantom{0}\cdot}}}.\]
First suppose $p^0\geq0$:
\begin{align*} 
	p^{\prime 0}&=\Lambda^0_{\hphantom{0}0} p^0 +\Lambda^0_{\hphantom{0}i}p^i\\
	&\geq \Lambda^0_{\hphantom{0}0} p^0 - \abs{\vec{\Lambda^{0}_{\hphantom{0}\cdot}}} \abs{\vec{p}}\\
	&= \abs{\Lambda^0_{\hphantom{0}0}} \abs{p^0} - \abs{\vec{\Lambda^{0}_{\hphantom{0}\cdot}}} \abs{\vec{p}}\\
	&\geq \abs{\vec{\Lambda^{0}_{\hphantom{0}\cdot}}}\left( \abs{p^0}-\abs{\vec{p}}\right)\geq 0
\end{align*}
Now suppose $p^0\leq0$:
\begin{align*} 
	p^{\prime 0}&=\Lambda^0_{\hphantom{0}0} p^0 +\Lambda^0_{\hphantom{0}i}p^i\\
	&\leq \Lambda^0_{\hphantom{0}0} p^0 + \abs{\vec{\Lambda^{0}_{\hphantom{0}\cdot}}} \abs{\vec{p}}\\
	&= -\abs{\Lambda^0_{\hphantom{0}0}} \abs{p^0} + \abs{\vec{\Lambda^{0}_{\hphantom{0}\cdot}}} \abs{\vec{p}}\\
	&\leq \abs{\Lambda^0_{\hphantom{0}0}}\left( -\abs{p^0}+\abs{\vec{p}}\right)\leq 0
\end{align*}

\subsubsection{\eq{2.5.12} \enquote{The delta function appears here because $\Psi_{k,\sigma}$ and $\Psi_{k^\prime,\sigma^\prime}$ are eigenstates of a Hermitian operator with eigenvalues $\tvec{k}$ and $\tvec{k}^\prime$, respectively.} \page{66}}\label{sususec:2_5_p66_1}
\fbox{\parbox{\linewidth}{{\color{red}Note:} From this point forward in this section 
		of the book the states considered have standard momentum $k^\mu$! \\
		(see Text in the Book preceding \eq{2.5.12})
}}
Suppose that for given values of $\tvec{k}$ and $\tvec{k}^\prime$ \[\innerproduct{\Psi_{k^\prime,\sigma^\prime}}{\Psi_{k,\sigma}}\neq 0\]
then from
\begin{align*} 
	&k^i\innerproduct{\Psi_{k^\prime,\sigma^\prime}}{\Psi_{k,\sigma}}\\
	&=\innerproduct{\Psi_{k^\prime,\sigma^\prime}}{P^i\Psi_{k,\sigma}}\\
	&=\innerproduct{P^i\Psi_{k^\prime,\sigma^\prime}}{\Psi_{k,\sigma}}\\
	&=k^{\prime i}\innerproduct{\Psi_{k^\prime,\sigma^\prime}}{\Psi_{k,\sigma}}
\end{align*}
follows \[k^i=k^{\prime i}.\]
On why there is only a 3 dimensional delta function showing up:


This is because both states are assumed to have standard momentum $k^\mu$ and therefore their 0-th component is completely fixed from their spatial components:
\[k^{\prime 0}=\pm \sqrt{\tvec{k}^{\prime 2}-k^2}\]
Choice between $+$ and $-$ comes from sign of $k^0$, i.e. $+$ for cases (a) and (c) of Table 2.1.
\subsubsection{\eq{2.5.13} \page{67}}
\begin{align*} 
	&\innerproduct{U(W)\Psi_{k^\prime,\sigma^\prime}}{U(W)\Psi_{k,\sigma}}\\
	&\overset{\eqc{2.5.8}}{=}\sum\limits_{\sigma^{\prime\prime}\sigma^{\prime\prime\prime}}\innerproduct{D_{\sigma^{\prime\prime} \sigma^{\prime}}\Psi_{k^\prime,\sigma^{\prime\prime}}}{D_{\sigma^{\prime\prime\prime} \sigma}\Psi_{k,\sigma^{\prime\prime\prime}}}\\
	&=\sum\limits_{\sigma^{\prime\prime}\sigma^{\prime\prime\prime}}\left(D_{\sigma^{\prime\prime} \sigma^{\prime}}\right)^\star D_{\sigma^{\prime\prime\prime} \sigma} \innerproduct{\Psi_{k^\prime,\sigma^{\prime\prime}}}{\Psi_{k,\sigma^{\prime\prime\prime}}}\\
	&\overset{\eqc{2.5.12}}{=}\sum\limits_{\sigma^{\prime\prime}}\left(D_{\sigma^{\prime\prime} \sigma^{\prime}}\right)^\star D_{\sigma^{\prime\prime} \sigma} \delta^{(3)}(\tvec{k}^\prime- \tvec{k})\\
	&\overset{\eqc{2.2.2}}{=}\innerproduct{\Psi_{k^\prime,\sigma^\prime}}{\Psi_{k,\sigma}}\\
	&\overset{\eqc{2.5.12}}{=} \delta_{\sigma^\prime \sigma} \delta^{(3)}(\tvec{k}^\prime- \tvec{k})
\end{align*}
From this it follows \[D^\dagger(W)=D^{-1}(W)\]

\subsubsection{\enquote{$\innerproduct{\Psi_{p^\prime,\sigma^\prime}}{\Psi_{p,\sigma}}=$ \dots} \page{67}}
{\color{red}Note:} What is meant by \enquote{arbitrary momenta} is that these momenta still have the standard momentum $k^\mu$, but now none of the states has exactly $k^\mu$ as its momentum.

First define \[k^\prime \coloneqq L^{-1}(p)p^\prime\]
with this we get:
\begin{align*} 
	&\innerproduct{\Psi_{p^\prime,\sigma^\prime}}{\Psi_{p,\sigma}}\\
	&\overset{\eqc{2.5.5}}{=}\innerproduct{\Psi_{p^\prime,\sigma^\prime}}{N(p)U(L(p))\Psi_{k,\sigma}}\\
	&=N(p)\innerproduct{U(L^{-1}(p))\Psi_{p^\prime,\sigma^\prime}}{\Psi_{k,\sigma}}\\
	&\overset{\eqc{2.5.11}}{=}N(p)\dfrac{N^\star\left(p^\prime\right)}{N^\star \left(k^\prime\right)}\\
	&\cdot\sum\limits_{\sigma^{\prime\prime}}
	D_{\sigma^{\prime\prime} \sigma^{\prime}}^\star\left(W\left(L^{-1}(p),p^\prime\right)\right)
	\innerproduct{\Psi_{k^\prime,\sigma^{\prime\prime}}}{\Psi_{k,\sigma}}\\
	&\overset{\eqc{2.5.12}}{=}N(p)N^\star\left(p^\prime\right) D_{\sigma \sigma^{\prime}}^\star\left(W\left(L^{-1}(p),p^\prime\right)\right) \delta^{(3)}(\tvec{k}^\prime- \tvec{k})
\end{align*}
Where in the last step we used 
\[\dfrac{\delta^{(3)}(\tvec{k}^\prime- \tvec{k})}{N^\star \left(k^\prime\right)}=\dfrac{\delta^{(3)}(\tvec{k}^\prime- \tvec{k})}{N^\star \left(k\right)}\overset{\eqc{2.5.5}}{=}\delta^{(3)}(\tvec{k}^\prime- \tvec{k})\]
This is valid as from \eq{2.5.5}  we know that $N(p)$ is implicitly dependent on $k^\mu$. 
This therefor fixes $p^2=k^2$ and the sign of $p^0$ (see end of \ref{sususec:2_5_p66_1}), s.t. \[N(p)=N(\tvec{p}).\]

Further we are allowed to use \eq{2.5.12} in the last step since $k^\prime$ also has  $k$ as its standard momentum:
\[k^\prime=L^{-1}(p)p^\prime=\underbrace{L^{-1}(p)L(p^\prime)}_{\eqqcolon L(k^\prime)}k\]

\subsubsection{\enquote{$W\left(L^{-1}(p),p\right)=1$} \page{67}}
First note \[L(k)=1=L^{-1}(k)\]
with this we get:
\begin{align*} 
	W\left(L^{-1}(p),p\right)&\overset{\eqc{2.5.10}}{=}L^{-1}\left(L^{-1}(p)p\right)L^{-1}(p)L(p)\\
	&=L^{-1}(k)=1
\end{align*}

\subsubsection{\enquote{So we see that the \textbf{invariant delta function} is \dots} \page{67}}
Invariant in this case means w.r.t. proper orthochronous Lorentz transformations, which can be interpreted as change of variables under the $\int \dd[4]{p}$ integral.

\subsubsection{\eq{2.5.24} \page{68}}

\begin{align*} 
	p^0&=L^0_{\hphantom{0}0} k^0 +L^0_{\hphantom{0}i} k^i\\
	&=\dfrac{\sqrt{\tvec{p}^2+M^2}}{M} M=\sqrt{\tvec{p}^2+M^2}\\
	p^i&=L^i_{\hphantom{i}0} k^0 + L^i_{\hphantom{i}j} k^j\\
	&=\dfrac{p_i}{\abs{\tvec{p}}}\sqrt{\dfrac{\tvec{p}^2+M^2}{M^2}-1} M\\
	&=\dfrac{p_i}{\abs{\tvec{p}}} \sqrt{\dfrac{\tvec{p}^2}{M^2}}M=p_i
\end{align*}

\subsubsection{\enquote{To see this, note that the boost \eq{2.5.24} may be expressed as $L(p)=R(\hat{\tvec{p}})B(\abs{\tvec{p}})R^{-1}(\hat{\tvec{p}})$} \page{68}}
First note that the columns and rows of the matrix $B(\abs{\tvec{p}})$ are counted in the order 1,2,3,0.


\begin{widetext}
	Let \[\hat{\tvec{p}}=\left(\begin{array}{c}
		\sin(\theta)\cos(\phi)\\
		\sin(\theta)\sin(\phi)\\
		\cos(\theta)
	\end{array}\right)\]
	then we have \begin{align*} R(-\hat{\tvec{p}})&=R_3(-\phi)R_2(\theta)=
		\left(\begin{array}{ccc}
			\cos(\phi)\cos(\theta)	& -\sin(\phi) & \cos(\phi)\sin(\theta) \\
			\sin(\phi)\cos(\theta)	& \cos(\phi) & 	\sin(\phi)\sin(\theta) \\
			-\sin(\theta)	& 0 & \cos(\theta)
		\end{array}\right)\end{align*} (see \ref{sususec:2_5_p73_1}) and 
	\begin{align*} R^{-1}(\hat{\tvec{p}})&=R_2(\theta)R_3(\phi)=
		\left(\begin{array}{ccc}
			\cos(\theta) & 0 &  -\sin(\theta)\\
			0	& 1 &  0\\
			\sin(\theta)& 0 & \cos(\theta)
		\end{array}\right)
		\left(\begin{array}{ccc}
			\cos(\phi)	& \sin(\phi) & 0 \\
			-\sin(\phi)	& \cos(\phi) & 0 \\
			0	& 0 & 1
		\end{array}\right)\\
		&=\left(\begin{array}{ccc}
			\cos(\theta) \cos(\phi)& \cos(\theta)\sin(\phi) &  -\sin(\theta)\\
			-\sin(\phi)	& \cos(\phi) &  0\\
			\sin(\theta) \cos(\phi)& \sin(\theta)\sin(\phi) & \cos(\theta)
		\end{array}\right)=R^\top (\hat{\tvec{p}})
	\end{align*}
	From \eq{2.5.24} we know
	\begin{align*} 
		L(p)&=\left(\begin{array}{cccc}
1+(\gamma-1)s_\theta^2c_\phi^2	& (\gamma -1)s_\theta c_\phi s_\theta s_\phi & 	(\gamma -1)s_\theta c_\phi c_\theta & s_\theta c_\phi\sqrt{\gamma^2-1} \\
(\gamma -1)s_\theta c_\phi s_\theta s_\phi&  1+(\gamma-1)s_\theta^2 s_\phi^2& (\gamma -1)s_\theta s_\phi c_\theta & s_\theta s_\phi\sqrt{\gamma^2-1} \\
(\gamma -1)s_\theta c_\phi c_\theta& (\gamma -1)s_\theta s_\phi c_\theta & 1+(\gamma-1)c_\theta^2 &c_\theta \sqrt{\gamma^2-1}  \\
s_\theta c_\phi\sqrt{\gamma^2-1}& s_\theta s_\phi\sqrt{\gamma^2-1} & c_\theta \sqrt{\gamma^2-1} &  \gamma
		\end{array}\right)
	\end{align*}
 where $s$ and $c$ are shorthand for $\sin$ and $\cos$. With this we can now check:
 \begin{align*} 
 	R(\hat{\tvec{p}})B(\abs{\tvec{p}})R^{-1}(\hat{\tvec{p}})&=\left(\begin{array}{cccc}
 	c_\phi c_\theta	& -s_\phi & c_\phi s_\theta &  0\\
 	s_\phi c_\theta	& c_\phi  & s_\phi s_\theta &  0\\
 	-s_\theta	& 0 &   c_\theta& 0 \\
 	0	& 0 & 0 & 1
 	\end{array}\right)
 	\left(\begin{array}{cccc}
 		1	& 0 & 0 &  0\\
 		0	& 1 & 0 &  0\\
 		0	& 0 & \gamma & \sqrt{\gamma^2-1} \\
 		0	& 0 & \sqrt{\gamma^2-1} & \gamma
 	\end{array}\right)
	 \left(\begin{array}{cccc}
	 	c_\phi c_\theta	& s_\phi c_\theta & -s_\theta &  0\\
	 	-s_\phi	& c_\phi  &0 &  0\\
	 	c_\phi s_\theta	& s_\phi s_\theta  &   c_\theta& 0 \\
	 	0	& 0 & 0 & 1
	 \end{array}\right)\\
 	&=
 	\left(\begin{array}{cccc}
 		c_\phi c_\theta	& -s_\phi & c_\phi s_\theta &  0\\
 		s_\phi c_\theta	& c_\phi  & s_\phi s_\theta &  0\\
 		-s_\theta	& 0 &   c_\theta& 0 \\
 		0	& 0 & 0 & 1
 	\end{array}\right)
 	\left(\begin{array}{cccc}
 		c_\phi c_\theta			& s_\phi c_\theta		  & -s_\theta &  0\\
 		-s_\phi					& c_\phi  				  &0 			&  0\\
 		\gamma c_\phi s_\theta	& \gamma s_\phi s_\theta  &   \gamma c_\theta& \sqrt{\gamma^2-1} \\
 		\sqrt{\gamma^2-1}c_\phi s_\theta & \sqrt{\gamma^2-1}s_\phi s_\theta & \sqrt{\gamma^2-1}c_\theta & \gamma
 	\end{array}\right)=L(p)
 \end{align*}
\end{widetext}

\subsubsection{\enquote{$W(\mathbf{R},p)=\mathbf{R}$} \page{69}}
To see this just substitute $R(\theta)$ back into the previous result.

\subsubsection{\eq{2.5.25} \page{70}}

\begin{align*} 
	-1=\left(Wt\right)^\mu \left(W t\right)_\mu &=\alpha^2+\beta^2+\zeta^2-(1+\zeta)^2\\
	\Leftrightarrow\alpha^2 + \beta^2&=2\zeta
\end{align*}

\subsubsection{\eq{2.5.26} \page{70}}
This is a Lorentz transformation, since
\begin{align*} 
	S^\top \left(\begin{array}{cccc}
		1&  0&  0&  0\\
		0&  1&  0&  0\\
		0&  0&  1&  0\\
		0&  0&  0& -1
	\end{array}\right) S &=
	\left(\begin{array}{cccc}
		1&  0&  \alpha&  \alpha\\
		0&  1&  \beta&  \beta\\
		-\alpha&  -\beta&  1-\zeta&  -\zeta\\
		\alpha&  \beta&  \zeta& 1+ \zeta
	\end{array}\right)\\
&\cdot\left(\begin{array}{cccc}
	1&  0&  -\alpha&  \alpha\\
	0&  1&  -\beta&  \beta\\
	\alpha&  \beta&  1-\zeta&  \zeta\\
	-\alpha&  -\beta&  \zeta& -1- \zeta
\end{array}\right)\\
&=\left(\begin{array}{cccc}
	1&  0&  0&  0\\
	0&  1&  0&  0\\
	0&  0&  1&  0\\
	0&  0&  0& -1
\end{array}\right)
\end{align*}

\subsubsection{\eqs{2.5.29/30} \page{70}}
\eq{2.5.30} is trivial since the rotations around the three axis satisfy this group multiplication and \[S(0,0)=\identity.\]
For \eq{2.5.29} explicit calculation shows it (see \todo) together with \[R(0)=\identity.\]

\subsubsection{\eq{2.5.31} \page{70}}
Explicit calculation shows \eq{2.5.31} (see \todo) and the invariance follows then immediately, since:
\begin{align*} 
	W^\prime SW^{\prime-1}=S^\prime \underbrace{R^\prime S R^{\prime-1}}_{=S^{\prime\prime}} S^{\prime-1}=S^{\prime\prime\prime}
\end{align*}

\subsubsection{\enquote{$W(\theta,\alpha,\beta)=1+\omega$} \page{71}}
From \eq{2.5.28} we immediately get for infinitesimal $\theta,\alpha,\beta$ :
\begin{align*} 
	(W(\theta,\alpha,\beta))^\mu_{\hphantom{\mu}\nu}&=\delta^\mu_\nu+
	\left(
	\begin{array}{cccc}
		0&  \theta&  -\alpha&  \alpha\\
		-\theta&  0&  -\beta&  \beta\\
		\alpha&  \beta&  0&  0\\
		\alpha&  \beta&  0& 0
	\end{array}
	\right)^\mu_{\hphantom{\mu}\nu}\\
	\Rightarrow \omega_{\mu \nu}&=\left(
	\begin{array}{cccc}
		0&  \theta&  -\alpha&  \alpha\\
		-\theta&  0&  -\beta&  \beta\\
		\alpha&  \beta&  0&  0\\
		-\alpha&  -\beta&  0& 0
	\end{array}
	\right)_{\mu\nu}
\end{align*}

\subsubsection{\eq{2.5.32} \page{71}}
\begin{align*} 
	U(1+\omega)&\overset{\eqc{2.4.3}}{=}1+\dfrac{1}{2}i \omega_{\rho \sigma} J^{\rho\sigma}\\
	&=1+i\theta J^{12}+i\alpha(-J^{13}+J^{10})+i\beta(-J^{23}+J^{20})\\
	&=1+i\theta J_3+i\alpha(J_2-K_1)+i\beta(-J_1-K_2)
\end{align*}

\subsubsection{\eqs{2.5.35/36/37} \page{71}}
From \eqs{2.4.18/19/20} we get:
\begin{align*} 
	\commutator{J_3}{A}&=\commutator{J_3}{J_2-K_1}\\
	&=\comm{J_3}{J_2}-\comm{J_3}{K_1}\\
	&=-i J_1-iK_2=iB\\
	\commutator{J_3}{B}&=-\comm{J_3}{J_1}-\commutator{J_3}{K_2}\\
	&=-iJ_2-i(-K_1)=-iA\\
	\commutator{A}{B}&=-\commutator{J_2}{J_1}-\commutator{J_2}{K_2}+\commutator{K_1}{J_1}+\commutator{K_1}{K_2}\\
	&=iJ_3-i J_3=0
\end{align*}
\subsubsection{\enquote{Since $A$ and $B$ are commuting Hermitian operators they can be simultaneously diagonalized by states $\Psi_{k,a,b}$} \page{71}}
This is valid, including the label $k$, since
\begin{align*} 
	\commutator{A}{P_1}\Psi_{k,a,b}&=(\commutator{J_2}{P_1}-\commutator{K_1}{P_1})\Psi_{k,a,b}\\
	&=(-iP_3+iP^0)\Psi_{k,a,b}=0\\
	\commutator{A}{P_2}\Psi_{k,a,b}&=(\commutator{J_2}{P_2}-\commutator{K_1}{P_2})\Psi_{k,a,b}\\
	&=0\\
	\commutator{A}{P_3}\Psi_{k,a,b}&=(\commutator{J_2}{P_3}-\commutator{K_1}{P_3})\Psi_{k,a,b}\\
	&=(iP_1)\Psi_{k,a,b}=0\\
	\commutator{A}{P^0}\Psi_{k,a,b}&=(\commutator{J_2}{P^0}-\commutator{K_1}{P^0})\Psi_{k,a,b}\\
	&=(-iP_1)\Psi_{k,a,b}=0\\
	\commutator{B}{P_1}\Psi_{k,a,b}&=(-\commutator{J_1}{P_1}-\commutator{K_2}{P_1})\Psi_{k,a,b}\\
	&=0\\
	\commutator{B}{P_2}\Psi_{k,a,b}&=(\commutator{J_1}{P_2}-\commutator{K_2}{P_2})\Psi_{k,a,b}\\
	&=(iP_3-iP^0)\Psi_{k,a,b}=0\\
	\commutator{B}{P_3}\Psi_{k,a,b}&=(\commutator{J_1}{P_3}-\commutator{K_2}{P_3})\Psi_{k,a,b}\\
	&=(-iP_2)\Psi_{k,a,b}=0\\
	\commutator{B}{P^0}\Psi_{k,a,b}&=(\commutator{J_1}{P^0}-\commutator{K_2}{P^0})\Psi_{k,a,b}\\
	&=(-iP_2)\Psi_{k,a,b}=0\\
\end{align*}

\subsubsection{\enquote{$\sigma$ gives the component of angular momentum in the direction of motion, or \emph{helicity}} \page{72}}
The derivation of \eq{2.5.26} starts among other conditions from the explicit form of $k$. Which results in \[\eqs{2.5.27/28}\rightarrow\eq{2.5.32}\rightarrow{2.5.39},\]
s.t. this is really connected to the direction of motion.

\subsubsection{\enquote{$U(W)\Psi_{k,\sigma}=\exp(i\theta\sigma)\Psi_{k,\sigma}$} \page{72}}
Use \eqs{2.5.38/39}.

\subsubsection{\eq{2.5.44} \page{73}}
With $B(u)$ from \eq{2.5.45} we get:
\begin{align*} 
	B\left(\dfrac{\abs{\tvec{p}}}{\kappa}\right) k&=
	\left(\begin{array}{cccc}
		1&  0&  0&  0\\
		0&  1&  0&  0\\
		0&  0&  \frac{\left(\frac{\abs{\tvec{p}}}{\kappa}\right)^2+1}{2\dfrac{\abs{\tvec{p}}}{\kappa}}&  \frac{\left(\frac{\abs{\tvec{p}}}{\kappa}\right)^2-1}{2\dfrac{\abs{\tvec{p}}}{\kappa}}\\
		0&  0&  \frac{\left(\frac{\abs{\tvec{p}}}{\kappa}\right)^2-1}{2\dfrac{\abs{\tvec{p}}}{\kappa}}& \frac{\left(\frac{\abs{\tvec{p}}}{\kappa}\right)^2+1}{2\dfrac{\abs{\tvec{p}}}{\kappa}}
	\end{array}\right)
	\left(
	\begin{array}{c}
		0\\
		0\\
		\kappa\\
		\kappa
	\end{array}
	\right)\\
	&=\abs{\tvec{p}}
	\left(
	\begin{array}{c}
		0\\
		0\\
		1\\
		1
	\end{array}
	\right)
\end{align*}

\subsubsection{\enquote{take $R(\hat{\tvec{p}})$ as a rotation by angle $\theta$ around the two-axis followed by a rotation by angle $\phi$ around the three-axis} \page{73}}\label{sususec:2_5_p73_1}
Let \[\hat{\tvec{p}}=\left(\begin{array}{c}
	\sin(\theta)\cos(\phi)\\
	\sin(\theta)\sin(\phi)\\
	\cos(\theta)
\end{array}\right)\]
then
\begin{align*} 
	R(\hat{\tvec{p}})&=R_3(-\phi)R_2(-\theta)\\
	&=
	\left(\begin{array}{ccc}
	\cos(\phi)	& -\sin(\phi) & 0 \\
	\sin(\phi)	& \cos(\phi) & 0 \\
	0	& 0 & 1
	\end{array}\right)
	\left(\begin{array}{ccc}
	\cos(\theta) & 0 &  \sin(\theta)\\
	0	& 1 &  0\\
	-\sin(\theta)& 0 & \cos(\theta)
	\end{array}\right)\\
	&=
	\left(\begin{array}{ccc}
		\cos(\phi)\cos(\theta)	& -\sin(\phi) & \cos(\phi)\sin(\theta) \\
		\sin(\phi)\cos(\theta)	& \cos(\phi) & 	\sin(\phi)\sin(\theta) \\
		-\sin(\theta)	& 0 & \cos(\theta)
	\end{array}\right),
\end{align*}
(With this sign convention everything is consistent, see definition of $R(\theta)$ after \eq{2.5.27} and compare \eq{2.5.47} to \eq{2.4.27})
with this we have \[\hat{\tvec{p}}=R(\hat{\tvec{p}}) \hat{e}_3. \]


\subsection{Space Inversion and Time-Reversal}\label{susec:2_6}

\subsubsection{\eqs{2.6.7-12} \page{76}}
\begin{align*} 
	\parity J_i \parity^{-1}&=\dfrac{1}{2}\varepsilon_{ijk} \parity J^{jk} \parity^{-1}\\
	&=\dfrac{1}{2}\varepsilon_{ijk}\left(-\delta^j_l\right)\left(-\delta^k_m\right)J^{lm} \\
	&=\dfrac{1}{2}\varepsilon_{ijk}J^{jk}=J_i\\
	\parity K_i \parity^{-1}&=\parity J^{0i} \parity^{-1}\\
	&=\delta^0_\mu \left(-\delta^j_\nu\right)J^{\mu\nu}\\
	&=-J^{0i}=-K_i\\
	\parity P_i \parity^{-1}&=\left(-\delta^i_\nu\right)P^\mu=-P_i\\
	\timereversal J_i \timereversal^{-1}&=\dfrac{1}{2}\varepsilon_{ijk} \timereversal J^{jk} \timereversal^{-1}\\
	&=-\dfrac{1}{2}\varepsilon_{ijk}\delta^j_l\delta^k_mJ^{lm} \\
	&=-\dfrac{1}{2}\varepsilon_{ijk}J^{jk}=-J_i\\
	\timereversal K_i \timereversal^{-1}&=\timereversal J^{0i} \timereversal^{-1}\\
	&=-\left(-\delta^0_\mu\right) \delta^j_\nu J^{\mu\nu}\\
	&=J^{0i}=K_i\\
	\timereversal P_i \timereversal^{-1}&=-\delta^i_\nu P^\mu=-P_i
\end{align*}

\subsubsection{\enquote{$\paritymatrix L(p)\paritymatrix^{-1}=L(\paritymatrix p)$} \page{77}}
From \eq{2.5.24} we immediately see:
\begin{align*} 
	\left(\paritymatrix L(p)\paritymatrix^{-1}\right)^i_{\hphantom{i}k}&=\left(L(p)\right)^i_{\hphantom{i}k}
	=\left(L(\paritymatrix p)\right)^i_{\hphantom{i}k}\\
	\left(\paritymatrix L(p)\paritymatrix^{-1}\right)^i_{\hphantom{i}0}&=-\left(L(p)\right)^i_{\hphantom{i}0}
	=\left(L(\paritymatrix p)\right)^i_{\hphantom{i}0}\\
	\left(\paritymatrix L(p)\paritymatrix^{-1}\right)^0_{\hphantom{0}0}&=\left(L(p)\right)^0_{\hphantom{0}0}
	=\left(L(\paritymatrix p)\right)^0_{\hphantom{0}0}
\end{align*}

\subsubsection{\enquote{Using \eq{2.6.14} again on the left, we see that the square-root factors cancel,\dots} \page{77}}
\begin{align*} 
	&\left(-J_1\pm i J_2\right)\zeta_\sigma \Psi_{k,-\sigma}=-\left(J_1\mp i J_2\right)\zeta_\sigma \Psi_{k,-\sigma}\\
	&\overset{\eqc{2.6.14}}{=}-\sqrt{(j\pm (-\sigma))(j\mp(-\sigma)+1)}\zeta_\sigma \Psi_{k,-\sigma\mp 1}\\
	&=-\sqrt{(j\mp\sigma)(j\pm\sigma+1)}\zeta_\sigma \Psi_{k,-\sigma\mp 1}
\end{align*}

\subsubsection{\enquote{The time-reversal phase $\zeta$ has no physical significance.} \page{78}}
This redefinition only works because \timereversal is \emph{anti-linear}.

\subsubsection{\enquote{$\timereversalmatrix L(p)\timereversalmatrix^{-1}=L(\paritymatrix p)$} \page{78}}
From \eq{2.5.24} we immediately see:
\begin{align*} 
	\left(\timereversalmatrix L(p)\timereversalmatrix^{-1}\right)^i_{\hphantom{i}k}&=\left(L(p)\right)^i_{\hphantom{i}k}
	=\left(L(\paritymatrix p)\right)^i_{\hphantom{i}k}\\
	\left(\timereversalmatrix L(p)\timereversalmatrix^{-1}\right)^i_{\hphantom{i}0}&=-\left(L(p)\right)^i_{\hphantom{i}0}
	=\left(L(\paritymatrix p)\right)^i_{\hphantom{i}0}\\
	\left(\timereversalmatrix L(p)\timereversalmatrix^{-1}\right)^0_{\hphantom{0}0}&=\left(L(p)\right)^0_{\hphantom{0}0}
	=\left(L(\paritymatrix p)\right)^0_{\hphantom{0}0}
\end{align*}

\subsubsection{\enquote{\parity yields a state with four-momentum \dots} \page{78}}\label{sususec:2_6_p78_1}
What is meant by this is:
\begin{align*} 
	P^i \parity \Psi_{k,\sigma}&\overset{\eqc{2.6.9}}{=} \parity \left(-P^i\right)\Psi_{k,\sigma}=\left(-\delta^i_3\kappa\right)\parity\Psi_{k,\sigma}\\
	H \parity \Psi_{k,\sigma}&\overset{\eqc{2.6.13}}{=} \parity H \Psi_{k,\sigma} =\kappa \parity\Psi_{k,\sigma}\\
	J_3 \parity \Psi_{k,\sigma}&\overset{\eqc{2.6.7}}{=} \parity J_3 \Psi_{k,\sigma} = \sigma \parity \Psi_{k,\sigma}
\end{align*}

\subsubsection{\eq{2.6.20} \page{79}}\label{sususec:2_6_p79_3}
We have \[R_2 = \left(
\begin{array}{cccc}
	-1 & 0 & 0 & 0 \\
	0 & 1 & 0 & 0 \\
	0 & 0 & -1 & 0 \\
	0 & 0 & 0 & 1
\end{array}
\right)=R_2^{-1}\]
from which we get (Is $U(R_2^{-1})=U^{-1}(R_2)$ ? \todo)
\begin{align*} 
	U^{-1}(R_2)J_3U(R_2)&=U^{-1}(R_2)J^{12}U(R_2)\\
	&\overset{\eqc{2.4.8}}{=}\left(R_2^{-1}\right)^1_{\hphantom{1}\mu}\left(R_2^{-1}\right)^2_{\hphantom{2}\nu} J^{\mu\nu}\\
	&=- J^{12}=-J_3\\
	U^{-1}(R_2)P^\nu U(R_2)&\overset{\eqc{2.4.9}}{=}\left(R_2^{-1}\right)^\nu_{\hphantom{\nu}\mu}P^\mu
\end{align*}
such that
\begin{align*} 
	J_3 U(R_2^{-1}) \parity \Psi_{k,\sigma}&= -U(R_2^{-1})J_3 \parity \Psi_{k,\sigma}\\
	&\overset{\ref{sususec:2_6_p78_1}}{=}-\sigma U(R_2^{-1})\parity \Psi_{k,\sigma}\\
	P^i U(R_2^{-1}) \parity \Psi_{k,\sigma}&\overset{\ref{sususec:2_6_p78_1}}{=}U(R_2^{-1}) \left(-\left(-\delta^i_3\kappa\right)\right) \parity \Psi_{k,\sigma}\\
	&=k^iU(R_2^{-1}) \parity \Psi_{k,\sigma}\\
	H U(R_2^{-1}) \parity \Psi_{k,\sigma}&\overset{\ref{sususec:2_6_p78_1}}{=} \kappa U(R_2^{-1}) \parity \Psi_{k,\sigma}\\
	&=k^0U(R_2^{-1}) \parity \Psi_{k,\sigma}.
\end{align*}
Further we get
\[R_2^{-1}\paritymatrix=\left(
\begin{array}{cccc}
	-1 & 0 & 0 & 0 \\
	0 & 1 & 0 & 0 \\
	0 & 0 & 1 & 0 \\
	0 & 0 & 0 & 1
\end{array}
\right).\]

\subsubsection{\enquote{\paritymatrix commutes with the rotation $R\left(\hat{\tvec{p}}\right)$} \page{79}}
This is true because $R\left(\hat{\tvec{p}}\right)$ only acts on space components non trivially, which all just get a \enquote{-} sign from \paritymatrix.

\subsubsection{\enquote{$\parity\Psi_{p,\sigma}=\sqrt{\dfrac{\kappa}{p^0}}\eta_\sigma U\left( R\left(\hat{\tvec{p}}\right)R_2  B\left(\dfrac{\abs{\tvec{p}}}{\kappa}\right)\right)\Psi_{k,-\sigma}$} \page{79}}\label{sususec:2_6_p79_1}
\begin{align*} 
	\parity\Psi_{p,\sigma}&\overset{\eqc{2.5.5}}{=}N(p)\parity U(L(p))\Psi_{k,\sigma}\\
	&\overset{\eqc{2.5.44}}{=}\sqrt{\dfrac{\kappa}{p^0}}\parity U\left(R\left(\hat{\tvec{p}}\right)B\left(\dfrac{\abs{\tvec{p}}}{\kappa}\right)\right)\Psi_{k,\sigma}\\
	&=\sqrt{\dfrac{\kappa}{p^0}}U\left(\paritymatrix R\left(\hat{\tvec{p}}\right)B\left(\dfrac{\abs{\tvec{p}}}{\kappa}\right)\right)\Psi_{k,\sigma}\\
	&=\sqrt{\dfrac{\kappa}{p^0}}U\left( R\left(\hat{\tvec{p}}\right)\paritymatrix B\left(\dfrac{\abs{\tvec{p}}}{\kappa}\right)\right)\Psi_{k,\sigma}\\
	&=\sqrt{\dfrac{\kappa}{p^0}}U\left( R\left(\hat{\tvec{p}}\right)R_2 R_2^{-1}\paritymatrix B\left(\dfrac{\abs{\tvec{p}}}{\kappa}\right)\right)\Psi_{k,\sigma}\\
	&=\sqrt{\dfrac{\kappa}{p^0}}U\left( R\left(\hat{\tvec{p}}\right)R_2  B\left(\dfrac{\abs{\tvec{p}}}{\kappa}\right)R_2^{-1}\paritymatrix\right)\Psi_{k,\sigma}\\
	&=\sqrt{\dfrac{\kappa}{p^0}}U\left( R\left(\hat{\tvec{p}}\right)R_2  B\left(\dfrac{\abs{\tvec{p}}}{\kappa}\right)\right)U\left(R_2^{-1}\right)\parity\Psi_{k,\sigma}\\
	&\overset{\eqc{2.6.20}}{=}\sqrt{\dfrac{\kappa}{p^0}}\eta_\sigma U\left( R\left(\hat{\tvec{p}}\right)R_2  B\left(\dfrac{\abs{\tvec{p}}}{\kappa}\right)\right)\Psi_{k,-\sigma}\\
\end{align*}

\subsubsection{\enquote{But a rotation of $\pm180^\circ$ around the three-axis reverses the sign of $J_2$,\dots} \page{79}}
Analogously to \ref{sususec:2_6_p78_1} ($2\leftrightarrow3$).

\subsubsection{\eq{2.6.22} \page{79}}
First note that \[B\left(\dfrac{\abs{\tvec{p}}}{\kappa}\right)\] and rotations around the three-axis commute (see \eq{2.5.45}).
\begin{align*} 
	\parity\Psi_{p,\sigma}&\overset{\ref{sususec:2_6_p79_1}}{=}\sqrt{\dfrac{\kappa}{p^0}}\eta_\sigma U\left( R\left(\hat{\tvec{p}}\right)R_2  B\left(\dfrac{\abs{\tvec{p}}}{\kappa}\right)\right)\Psi_{k,-\sigma}\\
	&\overset{\eq{2.6.21}}{=}\sqrt{\dfrac{\kappa}{p^0}}\eta_\sigma U\left( R\left(-\hat{\tvec{p}}\right)\right)\\
	&\cdot\exp(\pm i \pi J_3)U\left(  B\left(\dfrac{\abs{\tvec{p}}}{\kappa}\right)\right)\Psi_{k,-\sigma}\\
	&=\sqrt{\dfrac{\kappa}{p^0}}\eta_\sigma U\left( R\left(-\hat{\tvec{p}}\right)\right)\\
	&\cdot U\left(  B\left(\dfrac{\abs{\tvec{p}}}{\kappa}\right)\right)\exp(\pm i \pi J_3)\Psi_{k,-\sigma}\\
	&=\eta_\sigma \exp(\pm i \pi (-\sigma)) \sqrt{\dfrac{\kappa}{p^0}}U\left(L(\paritymatrix p)\right)\Psi_{k,-\sigma}\\
	&=\eta_\sigma \exp(\mp i \pi \sigma) \Psi_{\paritymatrix p,-\sigma}
\end{align*}

\subsubsection{\enquote{This peculiar change of sign in the operation of parity for mass-less particles of half-integer spin is due to the convention adopted in \eq{2.5.47} for the rotation used to define mass-less particles states of arbitrary momentum.} \page{79}}
This is based on the fact that the choice of $R\left(\hat{\tvec{p}}\right)$, which transforms the three axis into the unit vector $\hat{\tvec{p}}$, is \emph{not} unique. As mentioned in the text right after \eq{2.5.47}, one could always add an initial rotation around the three axis. This is also why the factor of \[\exp(\pm i \pi J_3)\] shows up in \eq{2.6.21}, despite $R\left(\hat{\tvec{p}}\right)R_2$ and $R\left(-\hat{\tvec{p}}\right)$ both transforming the three axis into the unit vector $-\hat{\tvec{p}}$.

\subsubsection{\enquote{\timereversal yields a state which has values \dots} \page{79}}\label{sususec:2_6_p79_2}
What is meant by this is:
\begin{align*} 
	P^i \timereversal \Psi_{k,\sigma}&\overset{\eqc{2.6.12}}{=} \timereversal \left(-P^i\right)\Psi_{k,\sigma}=\left(-\delta^i_3\kappa\right)\timereversal\Psi_{k,\sigma}\\
	H \timereversal \Psi_{k,\sigma}&\overset{\eqc{2.6.13}}{=} \timereversal H \Psi_{k,\sigma} =\kappa \timereversal\Psi_{k,\sigma}\\
	J_3 \timereversal \Psi_{k,\sigma}&\overset{\eqc{2.6.10}}{=} \timereversal \left(-J_3\right) \Psi_{k,\sigma} = -\sigma \timereversal \Psi_{k,\sigma}
\end{align*}

\subsubsection{\eq{2.6.23} \page{80}}
This is completely analog to \eq{2.6.20} when using \ref{sususec:2_6_p79_2} (see \ref{sususec:2_6_p79_3}).
Further we get
\[R_2^{-1}\timereversal=\left(
\begin{array}{cccc}
	-1 & 0 & 0 & 0 \\
	0 & 1 & 0 & 0 \\
	0 & 0 & -1 & 0 \\
	0 & 0 & 0 & -1
\end{array}
\right).\]

\subsubsection{\eq{2.6.24} \page{80}}\label{sususec:2_6_p80_1}
\begin{align*} 
	\timereversal\Psi_{p,\sigma}&\overset{\eqc{2.5.5}}{=}N(p)\timereversal U(L(p))\Psi_{k,\sigma}\\
	&\overset{\eqc{2.5.44}}{=}\sqrt{\dfrac{\kappa}{p^0}}\timereversal U\left(R\left(\hat{\tvec{p}}\right)B\left(\dfrac{\abs{\tvec{p}}}{\kappa}\right)\right)\Psi_{k,\sigma}\\
	&=\sqrt{\dfrac{\kappa}{p^0}}U\left(\timereversalmatrix R\left(\hat{\tvec{p}}\right)B\left(\dfrac{\abs{\tvec{p}}}{\kappa}\right)\right)\Psi_{k,\sigma}\\
	&=\sqrt{\dfrac{\kappa}{p^0}}U\left( R\left(\hat{\tvec{p}}\right)\timereversalmatrix B\left(\dfrac{\abs{\tvec{p}}}{\kappa}\right)\right)\Psi_{k,\sigma}\\
	&=\sqrt{\dfrac{\kappa}{p^0}}U\left( R\left(\hat{\tvec{p}}\right)R_2 R_2^{-1}\timereversalmatrix B\left(\dfrac{\abs{\tvec{p}}}{\kappa}\right)\right)\Psi_{k,\sigma}\\
	&=\sqrt{\dfrac{\kappa}{p^0}}U\left( R\left(\hat{\tvec{p}}\right)R_2  B\left(\dfrac{\abs{\tvec{p}}}{\kappa}\right)R_2^{-1}\timereversalmatrix\right)\Psi_{k,\sigma}\\
	&=\sqrt{\dfrac{\kappa}{p^0}}U\left( R\left(\hat{\tvec{p}}\right)R_2  B\left(\dfrac{\abs{\tvec{p}}}{\kappa}\right)\right)U\left(R_2^{-1}\right)\timereversal\Psi_{k,\sigma}\\
	&\overset{\eqc{2.6.23}}{=}\sqrt{\dfrac{\kappa}{p^0}}\zeta_\sigma U\left( R\left(\hat{\tvec{p}}\right)R_2  B\left(\dfrac{\abs{\tvec{p}}}{\kappa}\right)\right)\Psi_{k,\sigma}\\
\end{align*}

\subsubsection{\eq{2.6.25} \page{80}}
First note that \[B\left(\dfrac{\abs{\tvec{p}}}{\kappa}\right)\] and rotations around the three-axis commute (see \eq{2.5.45}).
\begin{align*} 
	\timereversal\Psi_{p,\sigma}&\overset{\ref{sususec:2_6_p80_1}}{=}\sqrt{\dfrac{\kappa}{p^0}}\zeta_\sigma U\left( R\left(\hat{\tvec{p}}\right)R_2  B\left(\dfrac{\abs{\tvec{p}}}{\kappa}\right)\right)\Psi_{k,\sigma}\\
	&\overset{\eq{2.6.21}}{=}\sqrt{\dfrac{\kappa}{p^0}}\zeta_\sigma U\left( R\left(-\hat{\tvec{p}}\right)\right)\\
	&\cdot\exp(\pm i \pi J_3)U\left(  B\left(\dfrac{\abs{\tvec{p}}}{\kappa}\right)\right)\Psi_{k,\sigma}\\
	&=\sqrt{\dfrac{\kappa}{p^0}}\zeta_\sigma U\left( R\left(-\hat{\tvec{p}}\right)\right)\\
	&\cdot U\left(  B\left(\dfrac{\abs{\tvec{p}}}{\kappa}\right)\right)\exp(\pm i \pi J_3)\Psi_{k,\sigma}\\
	&=\zeta_\sigma \exp(\pm i \pi \sigma) \sqrt{\dfrac{\kappa}{p^0}}U\left(L(\paritymatrix p)\right)\Psi_{k,\sigma}\\
	&=\zeta_\sigma \exp(\pm i \pi \sigma) \Psi_{\paritymatrix p,\sigma}
\end{align*}

\subsubsection{\enquote{\dots the total angular momentum $j$ of any state of this system would have to be a half-integer,\dots} \page{81}}
This is true as all spins and helicities except for one half integer spin/helicity would couple to a an integer angular momentum. And this would then couple with the remaining half integer spin/helicity to a half-integer total angular momentum $j$.


\subsection{Projective Representations}\label{susec:2_7}

\subsubsection{\eq{2.7.6} \page{83}}
From \eq{2.2.20} together with the modification of \eq{2.7.1} we have up to $\order{\theta^2,\bar{\theta}^2}$
\begin{align*} 
	&\left(1+i f_{ab}\bar{\theta}^a\theta^b\right)\\
	&\cdot\left[1+i\left(\theta^a+\bar{\theta}^a+f^a_{\hphantom{a} bc}\bar{\theta}^b\theta^c\right)t_a+
	\dfrac{1}{2}\left(\theta^b+\bar{\theta}^b\right)\left(\theta^c+\bar{\theta}^c\right)t_{bc}\right]\\
	&=\left[1+i\bar{\theta}^a t_a+\dfrac{1}{2}\bar{\theta}^b\bar{\theta}^c t_{bc}\right]\cdot\left[1+i{\theta}^a t_a+\dfrac{1}{2}{\theta}^b{\theta}^c t_{bc}\right]\\
	&=1+i{\theta}^a t_a+\dfrac{1}{2}{\theta}^b{\theta}^c t_{bc}+i\bar{\theta}^a t_a-\bar{\theta}^b t_b \theta^c t_c+\dfrac{1}{2}\bar{\theta}^b\bar{\theta}^c t_{bc}\\
	\Leftrightarrow &i f^a_{\hphantom{a} bc}\bar{\theta}^b\theta^c t_a + \dfrac{1}{2}\left(\theta^b \bar{\theta}^c+ \bar{\theta}^b\theta^c\right)t_{bc} +i f_{ab}\bar{\theta}^a\theta^b=-\bar{\theta}^b  \theta^c t_b t_c\\
	\Leftrightarrow&\bar{\theta}^b\theta^c\left[t_{bc}+ i f^a_{\hphantom{a} bc} t_a+t_b t_c+i f_{bc}\right]=0\\
	\Leftrightarrow& t_{bc}=- i f^a_{\hphantom{a} bc} t_a-t_b t_c-i f_{bc}
\end{align*}
Note that here the order of multiplication is adapted from \eq{2.2.18} and not from \eq{2.7.1}, which does not matter for the result since $\theta$ and $\bar{\theta}$ are dropped in the end. With this we now get the analog of \eq{2.2.22}:
\begin{align*} 
	&- i f^a_{\hphantom{a} bc} t_a-t_b t_c -i f_{bc}\\
	=t_{bc}&=t_{cb} = - i f^a_{\hphantom{a} cb} t_a-t_c t_b-i f_{cb}\\
	\Leftrightarrow\commutator{t_b}{t_c}&=i\left(f^a_{\hphantom{a} cb}-f^a_{\hphantom{a} bc}\right)t_a+i\left(f_{cb}-f_{bc}\right)\identity
\end{align*}

\subsubsection{\eq{2.7.12} \page{83}}
\begin{align*} 
	\commutator{\tilde{t}_b}{\tilde{t}_c}&=\commutator{t_b}{t_c}=iC^a_{\hphantom{a}bc} t_a+iC^a_{\hphantom{a}bc}\phi_a \identity=iC^a_{\hphantom{a}bc}\tilde{t}_a
\end{align*}

\subsubsection{\eqs{2.7.23/24/25} \page{85}}
Inserting \eqs{2.7.14-16} into \eq{2.7.20} we get:
\begin{align*}
	0&\overset{\eqc{2.7.20}}{=}\commutator{J^{\mu\nu}}{C^{\rho,\mu}}\\
	&+\commutator{P^\sigma}{\eta^{\nu\rho}P^\mu-\eta^{\mu\rho}P^\nu+C^{\rho,\mu\nu}}\\
	&+\commutator{P^\rho}{\eta^{\sigma\mu}P^\nu-\eta^{\sigma\nu}P^\mu+C^{\mu\nu,\sigma}}\\
	&=\eta^{\nu\rho}\commutator{P^\sigma}{P^\mu} - \eta^{\mu\rho} \commutator{P^\sigma}{P^\nu}\\
	&+\eta^{\sigma\mu}\commutator{P^\rho}{P^\nu}-\eta^{\sigma\nu}\commutator{P^\rho}{P^\mu}\\
	\overset{\eqc{2.7.16}}{\Rightarrow} 0&=\eta^{\nu\rho}C^{\mu,\sigma} - \eta^{\mu\rho} C^{\nu,\sigma}\\
	&+\eta^{\sigma\mu}C^{\nu,\rho}-\eta^{\sigma\nu}C^{\mu,\rho}
\end{align*}

Inserting \eqs{2.7.13-15} into \eq{2.7.21} we get:
\begin{align*}
	0&\overset{\eqc{2.7.21}}{=}\commutator{J^{\lambda\eta}}{\eta^{\nu\rho}P^\mu-\eta^{\mu\rho}P^\nu+C^{\rho,\mu\nu}}\\
	&+\commutator{P^\rho}{-\eta^{\nu \lambda}J^{\mu \eta}+\eta^{\mu\lambda}J^{\nu\eta}+\eta^{\eta \mu}J^{\lambda\nu}-\eta^{\eta \nu}J^{\lambda \mu}-C^{\lambda\eta,\mu\nu}}\\
	&+\commutator{J^{\mu\nu}}{\eta^{\rho\lambda}P^\eta-\eta^{\rho\eta}P^\lambda+C^{\lambda\eta,\rho}}\\
	&=\eta^{\nu\rho}\commutator{J^{\lambda\eta}}{P^\mu}-\eta^{\mu\rho}\commutator{J^{\lambda\eta}}{P^\nu}\\
	&-\eta^{\nu \lambda}\commutator{P^\rho}{J^{\mu \eta}}+\eta^{\mu\lambda}\commutator{P^\rho}{J^{\nu\eta}}\\
	&+\eta^{\eta \mu}\commutator{P^\rho}{J^{\lambda\nu}}-\eta^{\eta \nu}\commutator{P^\rho}{J^{\lambda \mu}}\\
	&+\eta^{\rho\lambda}\commutator{J^{\mu\nu}}{P^\eta}-\eta^{\rho\eta}\commutator{J^{\mu\nu}}{P^\lambda}
\end{align*}
Using \eqs{2.7.14/15} we finally get:
\todo


Rest analog \todo


\subsubsection{\eq{2.7.26} \page{85}}
\begin{align*}
	0&=4C^{\mu,\sigma} - \delta^\mu_\nu C^{\nu,\sigma}\\
	&-\delta^\sigma_\rho C^{\mu,\rho}+\eta^{\sigma\mu}C^{\nu,\rho}\eta_{\nu \rho}\\
	&=2 C^{\mu,\sigma}+0=2 C^{\mu,\sigma}
\end{align*}

\subsubsection{\eqs{2.7.27/28} \page{85}}
\begin{align*}
	0&=4 C^{\mu,\lambda \eta}-\delta^\mu_\nu C^{\nu,\lambda \eta}-\eta^{\mu \eta}C^{\rho,\lambda \nu}\eta_{\nu\rho}\\
	&+\eta^{\lambda\mu} C^{\rho,\eta \nu}\eta_{\nu\rho}+\delta^\lambda_\rho C^{\rho,\mu \eta}-\delta^\eta_\rho C^{\rho,\mu \lambda}\\
	&+\delta^\lambda_\nu C^{\eta,\mu\nu}-\delta^\eta_\nu C^{\lambda, \mu \nu}\\
	&=3C^{\mu,\lambda \eta} -\eta^{\mu \eta}C^{\rho,\lambda \nu}\eta_{\nu\rho}
	+\eta^{\lambda\mu} C^{\rho,\eta \nu}\eta_{\nu\rho}\\
	&=3\left(C^{\mu,\lambda \eta}-\eta^{\mu \eta}C^\lambda+\eta^{\lambda\mu}C^\eta\right)
\end{align*}

\subsubsection{\eqs{2.7.29/30} \page{85}}
First note \[C^{\rho\nu,\lambda\eta}\eta_{\nu \rho}=0\]
from the antisymmetry of $J^{\rho\sigma}$ in \eq{2.7.13}.
\begin{align*}
	0&=4C^{\mu\sigma,\lambda\eta}-\delta^\mu_\nu C^{\nu\sigma,\lambda\eta}-\eta^{\sigma\mu}C^{\rho\nu,\lambda\eta}\eta_{\nu \rho}+\delta^\sigma_\rho C^{\rho\mu,\lambda\eta}\\
	&+\eta^{\eta\mu}C^{\lambda\nu,\rho\sigma}\eta_{\nu \rho}-\eta^{\lambda \mu}C^{\eta \nu,\rho \sigma}\eta_{\nu \rho}-\delta^\lambda_\rho C^{\mu\eta,\rho\sigma}+\delta^\eta_\rho C^{\mu\lambda,\rho\sigma}\\
	&+\eta^{\sigma \lambda}C^{\rho\eta,\mu\nu}\eta_{\nu \rho}-\delta^\lambda_\nu C^{\sigma\eta,\mu\nu}-\delta^\eta_\nu C^{\lambda\sigma,\mu\nu}+\eta^{\eta \sigma} C^{\lambda\rho, \mu\nu }\eta_{\nu \rho}\\
	&=2C^{\mu\sigma,\lambda\eta}-2\eta^{\eta\mu}C^{\lambda\sigma}+2\eta^{\lambda \mu}C^{\eta \sigma}\\
	&-2\eta^{\sigma \lambda}C^{\eta\mu}+2\eta^{\eta \sigma} C^{\lambda \mu }\\
	&=2\left(C^{\mu\sigma,\lambda\eta}-\eta^{\eta\mu}C^{\lambda\sigma}+\eta^{\lambda \mu}C^{\eta \sigma}
	-\eta^{\sigma \lambda}C^{\eta\mu}+\eta^{\eta \sigma} C^{\lambda \mu }\right)
\end{align*}

\subsubsection{\eqs{2.7.33/34/35} \page{86}}
\begin{align*}
	&i\commutator{\tilde{J}^{\mu\nu}}{\tilde{J}^{\rho\sigma}}=i\commutator{J^{\mu\nu}}{J^{\rho\sigma}}\\
	&=\eta^{\nu\rho}J^{\mu\sigma}-\eta^{\mu\rho}J^{\nu\sigma}-\eta^{\sigma\mu}J^{\rho\nu}+\eta^{\sigma\nu}J^{\rho\mu}+C^{\rho\sigma,\mu\nu}\\
	&\overset{\eqc{2.7.29}}{=}\eta^{\nu\rho}J^{\mu\sigma}-\eta^{\mu\rho}J^{\nu\sigma}-\eta^{\sigma\mu}J^{\rho\nu}+\eta^{\sigma\nu}J^{\rho\mu}\\
	&+\eta^{\nu\rho}C^{\mu\sigma}-\eta^{\mu\rho}C^{\nu\sigma}+\eta^{\sigma\mu}C^{\nu\rho}-\eta^{\nu\sigma}C^{\mu\rho}\\
	&=\eta^{\nu\rho}J^{\mu\sigma}-\eta^{\mu\rho}J^{\nu\sigma}-\eta^{\sigma\mu}J^{\rho\nu}+\eta^{\sigma\nu}J^{\rho\mu}\\
	&+\eta^{\nu\rho}C^{\mu\sigma}-\eta^{\mu\rho}C^{\nu\sigma}-\eta^{\sigma\mu}C^{\rho\nu}+\eta^{\sigma\nu}C^{\rho\mu}\\
	&=\eta^{\nu\rho}\tilde{J}^{\mu\sigma}-\eta^{\mu\rho}\tilde{J}^{\nu\sigma}-\eta^{\sigma\mu}\tilde{J}^{\rho\nu}+\eta^{\sigma\nu}\tilde{J}^{\rho\mu}
\end{align*}

\begin{align*}
	&i\commutator{\tilde{J}^{\mu\nu}}{\tilde{P}^\rho}=i\commutator{J^{\mu\nu}}{P^\rho}\\
	&=\eta^{\nu\rho}P^\mu-\eta^{\mu\rho}P^\nu + C^{\rho,\mu\nu}\\
	&\overset{\eqc{2.7.27}}{=}\eta^{\nu\rho}P^\mu-\eta^{\mu\rho}P^\nu+\eta^{\rho\nu}C^\mu
	-\eta^{\rho\mu}C^\nu\\
	&=\eta^{\nu\rho}\tilde{P}^\mu-\eta^{\mu\rho}\tilde{P}^\nu
\end{align*}
\begin{align*}
	i\commutator{\tilde{P}^\mu}{\tilde{P}^\rho}=\commutator{P^\mu}{P^\rho}=0
\end{align*}

\subsubsection{\eq{2.7.42} \page{87}}
This does not set the overall phase completely because 
\[1=\det(\exp(i\theta)\lambda)=\exp(i2\theta)\det(\lambda)=\exp(i2\theta)\]
\dots so for $\theta=\pi$, $\exp(i\theta)\lambda=-\lambda$ satisfies the condition if $\lambda$ satisfies it. 

\subsubsection{\enquote{The group elements depend on $4-1=3$ complex parameters,\dots} \page{87}}
This can easily been seen by calculating the determinant of a general $2\times2$ complex matrix:
\begin{align*}
	\det(\begin{array}{cc}
		a&  b\\
		c& d
	\end{array})&=ad-bc=1\\
\overset{d\neq0}{\Rightarrow} a&=\dfrac{1+bc}{d}\\
\overset{d=0}{\Rightarrow} b&=-\dfrac{1}{c}
\end{align*}

\subsubsection{\enquote{\dots produces a Lorentz transformation $\Lambda(\lambda(\theta))$ which is just a rotation by an angle $\theta$ around the three-axis,\dots} \page{87}}
\begin{widetext}
	\begin{align*}
		&\lambda(\theta) v \lambda^\dagger (\theta)\\
		&=\left(\begin{array}{cc}
			\exp(i\dfrac{\theta}{2})& 0 \\
			0& \exp(-i\dfrac{\theta}{2})
		\end{array}\right)
		\left(\begin{array}{cc}
			V^0+V^3&V^1-iV^2  \\
			V^1+iV^2&V^0-V^3 
		\end{array}\right)
		\left(\begin{array}{cc}
			\exp(-i\dfrac{\theta}{2})& 0 \\
			0& \exp(i\dfrac{\theta}{2})
		\end{array}\right)\\
	&=\left(\begin{array}{cc}
		\exp(i\dfrac{\theta}{2})& 0 \\
		0& \exp(-i\dfrac{\theta}{2})
	\end{array}\right)
	\left(\begin{array}{cc}
		\exp(-i\dfrac{\theta}{2})(V^0+V^3)&\exp(i\dfrac{\theta}{2})(V^1-iV^2)  \\
		\exp(-i\dfrac{\theta}{2})(V^1+iV^2)&\exp(i\dfrac{\theta}{2})(V^0-V^3) 
	\end{array}\right)\\
	&=\left(\begin{array}{cc}
		V^0+V^3&\exp(i{\theta})(V^1-iV^2)  \\
		\exp(-i{\theta})(V^1+iV^2)&V^0-V^3
	\end{array}\right)
	\end{align*}
	From this we can read of:
	\begin{align*}
		V^{\prime 0} &= V^0\\
		V^{\prime 1} &= \Re{\exp(i{\theta})(V^1-iV^2)}=\Re{(\cos(\theta)+i\sin(\theta))(V^1-iV^2)}\\
		&=\cos(\theta)V^1+\sin(\theta)V^2\\
		V^{\prime 2} &= \Im{\exp(-i{\theta})(V^1+iV^2)}=\Im{(\cos(\theta)-i\sin(\theta))(V^1+iV^2)}\\
		&=-\sin(\theta)V^1+\cos(\theta)V^2
	\end{align*}
\end{widetext}

\subsubsection{\enquote{\dots $\det(\exp(h))=\exp(\tr(h))$ is real and positive} \page{88}}
This is true because eigenvalues of an hermitian matrix are real, s.t.:

\[\exp(tr(h))=\exp(\sum\limits_{i}e_i)>0\]

\subsubsection{\enquote{\dots with $d,e,f,g$ subject to the single non-linear constraint $d^2+e^2+f^2+g^2=1$,\dots} \page{88}}
\[u^\dagger u=1\] and \[\det(u)=1\] yield the same constraint.

\subsubsection{\enquote{because $\exp(h)$ is always positive} \page{89}}
The eigenvalues of $\exp(h)$ are positive, since
\[\exp(h)=\exp(u \diagm{e_i} u^\dagger)=u \diagm{\exp(e_i)} u^\dagger\]
and $e_i \in \mathbb{R}$.

\subsubsection{\enquote{$\left[U(\Lambda)U(\bar{\Lambda})U^{-1}(\Lambda\bar{\Lambda})\right]^2=\identity$} \page{89}}
This follows from the discussion in Appendix B, more precisely see \ref{sususec:2_B_p98_1}, because a contraction of the double loop to a point is possible.

\subsubsection{\enquote{These two cases correspond to the two irreducible representations of the first homotopy group $Z_2$} \page{89}}
These are the trivial representation \[1\rightarrow1\qquad-1\rightarrow1\] and the faithful representation \[1\rightarrow1\qquad-1\rightarrow-1.\]

\subsubsection{\enquote{We must not mix states of integer and half-integer spin.} \page{89}}
Because they are different representations of $Z_2$ or because some of their loops are contractable to a point compare Superselection rule in Section $2.2$.
\todo

\subsubsection{\enquote{\dots , so the factor $\exp(4\pi i \sigma)$ must be \textbf{unity}, and hence $\sigma$ must be an integer or half-integer.} \page{90}}
This can be seen from the transformation behavior of massless one particle states \eq{2.5.42} and 
\begin{align*}
	\identity \Psi_{p,\sigma} &=\left[U(R(2\pi))\right]^2 \Psi_{p,\sigma}\\
	&=\left(\exp(i\sigma 2\pi)\right)^2\Psi_{p,\sigma}\\
	&=\exp(i\sigma 4\pi)\Psi_{p,\sigma}
\end{align*}\todo

\subsection{The Symmetry Representation Theorem}\label{susec:2_A}

\subsubsection{\enquote{But $\innerproduct{\Psi_k^\prime}{\Psi_k^\prime}$ is automatically \textbf{real and positive}} \page{91}}
This follows immediately from \eq{2.1.1}.

\subsubsection{\enquote{From \eq{2.A.1} we have $\abs{c_{kk}}=\abs{c_{k1}}=\frac{1}{\sqrt{2}}$ and for $l\neq k$ and $l\neq 1$: $c_{kl}=0$} \page{91}}

\begin{align*}
	\abs{c_{kl^\prime}}^2&\overset{\eqc{2.A.3}}{=}\abs{\sum\limits_{l}c_{kl}^\star \innerproduct{\Psi_l^\prime}{\Psi_{l^\prime}^\prime}}^2=\abs{\innerproduct{\Upsilon_k^\prime}{\Psi_{l^\prime}^\prime}}^2\\
	&\overset{\eqc{2.A.1}}{=}\abs{\innerproduct{\Upsilon_k}{\Psi_{l^\prime}}}^2=\begin{cases}
		\frac{1}{2} &\text{for } l^\prime=1,k\\
		0&\text{for } l^\prime\neq1,k\end{cases}
\end{align*}

\subsubsection{\eq{2.A.10} \page{92}}
\begin{align*} 
	\abs{C_k}^2+\abs{C_1}^2+2\Re(C_k C_1^\star)&=\abs{C_k+C_1}^2\\
	\overset{\eqc{2.A.9}}{=}\abs{C_k^\prime+C_1^\prime}^2=\abs{C_k^\prime}^2+\abs{C_1^\prime}^2&+2\Re(C_k^\prime C_1^{\prime\star})\\
	\overset{\eq{2.A.8}}{\Rightarrow} \Re(C_k C_1^\star)&=\Re(C_k^\prime C_1^{\prime\star})\\
	\overset{\eq{2.A.8}}{\Rightarrow} \Re\left(\dfrac{C_k}{C_1}\right)&=\Re\left(\dfrac{C_k^\prime}{C_1^\prime}\right)
\end{align*}

\subsubsection{\eq{2.A.11} \page{92}}
\begin{align*} 
	\left\lbrace\Re\left(\dfrac{C_k}{C_1}\right)\right\rbrace^2+\left\lbrace\Im\left(\dfrac{C_k}{C_1}\right)\right\rbrace^2&=\abs{\dfrac{C_k}{C_1}}^2\\
	\overset{\eq{2.A.8}}{=}\abs{\dfrac{C_k^\prime}{C_1^\prime}}^2=\left\lbrace\Re\left(\dfrac{C_k^\prime}{C_1^\prime}\right)\right\rbrace^2&+\left\lbrace\Im\left(\dfrac{C_k^\prime}{C_1^\prime}\right)\right\rbrace^2\\
	\overset{\eq{2.A.10}}{\Rightarrow} \Im\left(\dfrac{C_k}{C_1}\right)&=\pm\Im\left(\dfrac{C_k^\prime}{C_1^\prime}\right)
\end{align*}

\subsubsection{\enquote{This is only possible if $\Re\left(\frac{C_k}{C_1}\frac{C_l^\star}{C_1^\star}\right)=\Re\left(\frac{C_k}{C_1}\frac{C_l}{C_1}\right)$ or, in other words, if $\Im\left(\frac{C_k}{C_1}\right)\Im\left(\frac{C_l}{C_1}\right)=0$} \page{93}}

Define 
\begin{align*} 
	a\coloneqq \dfrac{C_k}{C_1}\\
	b\coloneqq \dfrac{C_l}{C_1}
\end{align*}
With this we have
\begin{align} 
	\abs{1+a+b^\star}^2&=\abs{1+a+b}^2\\
	\Leftrightarrow 1 + a^\star+b+a&+\abs{a}^2+a b+b^\star+b^\star a^\star+\abs{b}^2\\
	=1 + a^\star+b^\star+a&+\abs{a}^2+a b^\star+b+b a^\star+\abs{b}^2\\
	\Leftrightarrow a b+b^\star a^\star&=a b^\star+b a^\star \label{eq_2_A_1}
\end{align}
And further rewriting yields
\begin{align*} 
	\Re\left(\dfrac{C_k}{C_1}\dfrac{C_l}{C_1}\right)&=\Re(a b)\overset{\ref{eq_2_A_1}}{=}\Re(a b^\star)=\Re\left(\dfrac{C_k}{C_1}\dfrac{C_l^\star}{C_1^\star}\right)\\
	\Im\left(\dfrac{C_k}{C_1}\right)\Im\left(\dfrac{C_l}{C_1}\right)&=\Im(a)\Im(b)\\
	&=-\dfrac{1}{4}(ab - a b^\star-a^\star b + a^\star b^\star)\overset{\ref{eq_2_A_1}}{=}0
\end{align*}

\subsubsection{\enquote{Then the invariance of transition probabilities requires that $\abs{\sum\limits_{k}B_k^\star A_k}^2=\abs{\sum\limits_{k}B_k A_k}^2$} \page{93}}\label{sususec:2_A_p93_1}
\begin{align*} 
	\abs{\sum\limits_{k}B_k^\star A_k}^2\overset{\eq{2.A.2}}{=}&\abs{\sum\limits_{k l}B_k^\star A_l \innerproduct{\Psi_k}{\Psi_l}}^2\\
	=&\abs{\innerproduct{\sum\limits_{k}B_k \Psi_k}{\sum\limits_{l}A_l \Psi_l}}^2\\
	\overset{\eq{2.A.1}}{=}&\abs{\innerproduct{U\left(\sum\limits_{k}B_k \Psi_k\right)}{U\left(\sum\limits_{l}A_l \Psi_l\right)}}^2\\
	=&\abs{\innerproduct{\sum\limits_{k}B_k^\star U\Psi_k}{\sum\limits_{l}A_l U\Psi_l}}^2\\
	=&\abs{\sum\limits_{k l}B_k A_l \innerproduct{U\Psi_k}{U\Psi_l}}^2\\
	\overset{\eq{2.A.3}}{=}&\abs{\sum\limits_{k}B_k A_k}^2
\end{align*}

\subsubsection{\eq{2.A.16}\page{94}}
\begin{align*} 
	&\sum\limits_{k l}\Im(B_k^\star B_l) \Im(A_k^\star A_l)\\
	&=\Im\left(\sum\limits_{k l}\Im(B_k^\star B_l) A_k^\star A_l\right)\\
	&=\dfrac{1}{2 i}\left[\sum\limits_{k l}\Im(B_k^\star B_l) A_k^\star A_l- \complexc\right]\\
	&=\dfrac{1}{2 i}\left[\sum\limits_{k l}\dfrac{1}{2i}(B_k^\star B_l-B_k B_l^\star) A_k^\star A_l- \complexc\right]\\
	&=\dfrac{1}{2 i}\left[\dfrac{1}{2i}\left(\sum\limits_{k l}B_k^\star B_l A_k^\star A_l- \sum\limits_{k l}B_k B_l^\star A_k^\star A_l\right)- \complexc\right]\\
	&=\dfrac{1}{2 i}\left[\dfrac{1}{2i}\left(\abs{\sum\limits_{k}B_k A_k}^2-\abs{\sum\limits_{k}B_k^\star A_k}^2\right)- \complexc\right]\\
	&=\dfrac{1}{2 i}\left[\dfrac{1}{i}\left(\abs{\sum\limits_{k}B_k A_k}^2-\abs{\sum\limits_{k}B_k^\star A_k}^2\right)\right]\\
	&\overset{\ref{sususec:2_A_p93_1}}{=}0
\end{align*}

\subsubsection{\enquote{However, for any pair of such state-vectors, with neither $A_k$ nor $B_k$ \textbf{all of the same phase}} \page{94}}

If they were all of the same phase then \[\forall k,l: \Im{A_k^\star A_l}=0\]
or \[\forall k,l: \Im{B_k^\star B_l}=0\]
See Footnote j, for why this is relevant.

\subsubsection{\enquote{We have thus shown that for a given symmetry transformation $T$ either all state-vectors satisfy \eq{2.A.14} or else they all satisfy \eq{2.A.15}} \page{94}}
Preceding this a contradiction was derived from the assumption that \eq{2.A.14} applies for a state-vector \[\sum\limits_{k}A_k\Psi_k\] while \eq{2.A.15} applies for a state vector \[\sum\limits_{k}B_k\Psi_k.\]
Such that the statement is obvious.


\subsection{Group Operators and Homotopy Classes}\label{susec:2_B}
\subsubsection{\eq{2.B.7} \page{97}}
Taylor evolving \eq{2.B.6} in $\theta^c_3$ up to $\order{\theta_3^2}$ yields:
\begin{align*} 
	&f^a(\theta_2,\theta_1)+\left[h^{-1}\right]^a_{\hphantom{a}c}\left(f(\theta_2,\theta_1)\right)\theta_3^c\\
	&=f^a(0,f(\theta_2,\theta_1))+\left[\partialderivative{f^a(\bar{\theta},f(\theta_2,\theta_1))}{\bar{\theta}^c}\right]_{\bar{\theta}=0}\theta_3^c\\
	&=f^a(\theta_3,f(\theta_2,\theta_1))\\
	&\overset{\eqc{2.B.6}}{=}f^a(f(\theta_3,\theta_2),\theta_1)\\
	&=f^a\left(f(0,\theta_2)+\left[\partialderivative{f^a(\bar{\theta},\theta_2)}{\bar{\theta}^c}\right]_{\bar{\theta}=0}\theta_3^c,\theta_1\right)\\
	&=f^a(\theta_2,\theta_1)+\left[\partialderivative{f^a(\bar{\theta},\theta_1)}{\bar{\theta}^b}\right]_{\bar{\theta}=\theta_2}\left[\partialderivative{f^b(\bar{\theta},\theta_2)}{\bar{\theta}^c}\right]_{\bar{\theta}=0}\theta_3^c\\
	&=f^a(\theta_2,\theta_1)+\left[\partialderivative{f^a(\bar{\theta},\theta_1)}{\bar{\theta}^b}\right]_{\bar{\theta}=\theta_2}\left[h^{-1}\right]^b_{\hphantom{b}c}\left(\theta_2\right)\theta_3^c
\end{align*}
Equating coefficients of $\theta^c_3$, we get:
\begin{align*} 
	\left[h^{-1}\right]^a_{\hphantom{a}c}\left(f(\theta_2,\theta_1)\right)&=\left[\partialderivative{f^a(\bar{\theta},\theta_1)}{\bar{\theta}^b}\right]_{\bar{\theta}=\theta_2}\left[h^{-1}\right]^b_{\hphantom{b}c}\left(\theta_2\right)\\
	\Leftrightarrow h^c_{\hphantom{c}b}\left(\theta_2\right)&=\left[\partialderivative{f^a(\bar{\theta},\theta_1)}{\bar{\theta}^b}\right]_{\bar{\theta}=\theta_2}h^c_{\hphantom{c}a}\left(f(\theta_2,\theta_1)\right)
\end{align*}

\subsubsection{\enquote{Along the second segment the differential equation \eq{2.B.2} for $U_\mathcal{P}(s)$ is thus the same as the differential equation for $U_{\theta_2}(2s-1)$.} \page{97}}

In the second segment $\left(\dfrac{1}{2}\leq s \leq 1\right)$ the differential equation for $U_\mathcal{P}(s)$ reads:
\begin{align*} 
	&\derivative{U_\mathcal{P}(s)}{s}\\
	&=i t_c U_\mathcal{P}(s) h^c_{\hphantom{c}a}\left(\Theta_\mathcal{P}(s)\right)\derivative{\Theta_\mathcal{P}^c(s)}{s}\\
	&=i t_c U_\mathcal{P}(s) h^c_{\hphantom{c}a}\left(f\left(\Theta_{\theta_2}(2s-1),\theta_1\right)\right)\derivative{f^c\left(\Theta_{\theta_2}(2s-1),\theta_1\right)}{s}\\
	&=i t_c U_\mathcal{P}(s) h^c_{\hphantom{c}a}\left(f\left(\Theta_{\theta_2}(2s-1),\theta_1\right)\right)\left[\partialderivative{f^a(\bar{\theta},\theta_1)}{\bar{\theta}^b}\right]_{\bar{\theta}=\Theta_{\theta_2}(2s-1)}\\
	&\cdot\derivative{\Theta_{\theta_2}^b(2s-1)}{s}\\
	&\overset{\eqc{2.B.7}}{=}i t_c U_\mathcal{P}(s)  h^c_{\hphantom{c}b}\left(\Theta_{\theta_2}(2s-1)\right)\derivative{\Theta_{\theta_2}^b(2s-1)}{s}
\end{align*}
Which is just the differential equation for $U_{\theta_2}(2s-1)$:
\begin{align*} 
	&\derivative{U_{\theta_2}(2s-1)}{s}=\left[\derivative{U_{\theta_2}(t)}{t}\right]_{t=2s-1}2\\
	&=i t_c U_{\theta_2}(2s-1)  h^c_{\hphantom{c}b}\left(\Theta_{\theta_2}(2s-1)\right)\left[\derivative{\Theta_{\theta_2}^b(t)}{t}\right]_{t=2s-1}2\\
	&=i t_c U_{\theta_2}(2s-1)  h^c_{\hphantom{c}b}\left(\Theta_{\theta_2}(2s-1)\right)\derivative{\Theta_{\theta_2}^b(2s-1)}{s}
\end{align*}

\subsubsection{\eq{2.B.9} \page{98}}
First note
\begin{align} 
	\derivative{U^{-1}}{s}&=\left(\derivative{U}{s}\right)^\dagger\overset{\eqc{2.B.2}}{=}-i\left(t_a U\right)^\dagger h^a_{\hphantom{a}b}\derivative{\Theta^b}{s}\\
	&=-iU^{-1} t_a h^a_{\hphantom{a}b}\derivative{\Theta^b}{s},\label{eq_2_B_1}
\end{align}
and 
\begin{align} 
	&\derivative{}{s}\left(U^{-1} t_a U h^a_{\hphantom{a}b}\right)\\
	&=\derivative{U^{-1}}{s}t_e U h^e_{\hphantom{e}b}+U^{-1} t_e \derivative{U}{s}h^e_{\hphantom{e}b}+U^{-1} t_a U\derivative{h^a_{\hphantom{a}b}}{s}\\
	&\overset{\eqc{2.B.2}}{=}-iU^{-1} t_d h^d_{\hphantom{d}c}\derivative{\Theta^c}{s}t_e U h^e_{\hphantom{e}b}\\
	&+U^{-1} t_e it_d U h^d_{\hphantom{d}c}\derivative{\Theta^c}{s}h^e_{\hphantom{e}b}\\
	&+U^{-1} t_a U h^a_{\hphantom{a}b,c} \derivative{\Theta^c}{s}\\
	&=i\derivative{\Theta^c}{s}h^d_{\hphantom{d}c}h^e_{\hphantom{e}b}U^{-1}\comm{t_e}{t_d}U+U^{-1} t_a U\derivative{\Theta^c}{s}h^a_{\hphantom{a}b,c}\\
	&\overset{\eqc{2.2.22}}{=}U^{-1} t_a U\derivative{\Theta^c}{s}\left(ih^d_{\hphantom{d}c}h^e_{\hphantom{e}b}iC^a_{\hphantom{a}ed}+h^a_{\hphantom{a}b,c}\right).\label{eq_2_B_2}
\end{align}
By using \eq{2.2.22} we are making use of condition (a) of the Theorem.
With this we get:
\begin{align*} 
	&\derivative{U^{-1}\delta U}{s}\\
	&=\derivative{U^{-1}}{s}\delta U + U^{-1}\derivative{\delta U}{s}\\
	&\overset{\ref{eq_2_B_1}}{=}-iU^{-1} t_a h^a_{\hphantom{a}b}\derivative{\Theta^b}{s} \delta U + U^{-1}\derivative{\delta U}{s}\\
	&=i U^{-1} t_a U h^a_{\hphantom{a}c,b}\left(\delta\Theta^b\right)\derivative{\Theta^c}{s}\\
	&+i U^{-1} t_a U h^a_{\hphantom{a}b}\derivative{\delta\Theta^b}{s}  \\
	&=i U^{-1} t_a U \left(\delta\Theta^b\right)\derivative{\Theta^c}{s}h^a_{\hphantom{a}c,b}\\
	&+\derivative{}{s}\left(iU^{-1} t_a U h^a_{\hphantom{a}b}\delta\Theta^b\right)-i\left(\delta\Theta^b\right)\derivative{}{s}\left(U^{-1} t_a U h^a_{\hphantom{a}b}\right)\\
	&\overset{\ref{eq_2_B_2}}{=}i U^{-1} t_a U \left(\delta\Theta^b\right)\derivative{\Theta^c}{s}\left(h^a_{\hphantom{a}c,b}-\left(-h^d_{\hphantom{d}c}h^e_{\hphantom{e}b}C^a_{\hphantom{a}ed}+h^a_{\hphantom{a}b,c}\right)\right)\\
	&+\derivative{}{s}\left(iU^{-1} t_a U h^a_{\hphantom{a}b}\delta\Theta^b\right)\\
	&=i U^{-1} t_a U \left(\delta\Theta^b\right)\derivative{\Theta^c}{s}\left(h^a_{\hphantom{a}c,b}+h^d_{\hphantom{d}c}h^e_{\hphantom{e}b}C^a_{\hphantom{a}ed}-h^a_{\hphantom{a}b,c}\right)\\
	&+\derivative{}{s}\left(iU^{-1} t_a U h^a_{\hphantom{a}b}\delta\Theta^b\right)
\end{align*}
Where we inserted the expression for $\derivative{\delta U}{s}$ in the third step.

\subsubsection{\eq{2.B.10} \page{98}}
Taylor evolving \eq{2.B.6} in $\theta_3, \theta_2$ up to $\order{\theta_3^3, \theta_2^3}$ yields:
\begin{align*} 
	&f^a(0,\theta_1)+\left[h^{-1}\right]^a_{\hphantom{a}b}(\theta_1)\left(\theta_3^b+\theta_2^b+f^b_{\hphantom{b}ec} \theta_3^e\theta_2^c\right)\\
	&=f^a(0,\theta_1)+\left[\partialderivative{f^a(\bar{\theta},\theta_1)}{\bar{\theta}^b}\right]_{\bar{\theta}=0}\left(\theta_3^b+\theta_2^b+f^b_{\hphantom{b}ec} \theta_3^e\theta_2^c\right)\\
	&=f^a\left(\theta_3+\theta_2+f^\cdot_{\hphantom{b}ec} \theta_3^e\theta_2^c,\theta_1\right)\\
	&\overset{\eqc{2.2.19}}{=}f^a(f(\theta_3,\theta_2),\theta_1)\\
	&\overset{\eqc{2.B.6}}{=}f^a(\theta_3,f(\theta_2,\theta_1))\\
	&=f^a\left(\theta_3, f(0,\theta_1)+\left[\partialderivative{f^a(\bar{\theta},\theta_1)}{\bar{\theta}^e}\right]_{\bar{\theta}=0}\theta_2^e\right)\\
	&=f^a\left(\theta_3, \theta_1+\left[h^{-1}\right]^\cdot_{\hphantom{a}e}(\theta_1)\theta_2^e\right)\\
	&=f^a(0,\theta_1)+\left[\partialderivative{f^a(\bar{\theta},\theta_1)}{\bar{\theta}^b}\right]_{\bar{\theta}=0}\theta_3^b\\
	&+\left[\partialderivative{f^a(0,\bar{\theta})}{\bar{\theta}^b}\right]_{\bar{\theta}=\theta_1}\left[h^{-1}\right]^b_{\hphantom{b}e}(\theta_1)\theta_2^e\\
	&+\dfrac{1}{2}\left[\partialderivative{}{\tilde{\theta}^c}\left[\partialderivative{f^a(\bar{\theta},\tilde{\theta})}{\bar{\theta}^b}\right]_{\bar{\theta}=0}\right]_{\tilde{\theta}=\theta_1}
	\theta_3^b	\left[h^{-1}\right]^c_{\hphantom{c}e}(\theta_1)\theta_2^e\\
	&+\dfrac{1}{2}\left[\partialderivative{}{\bar{\theta}^b}\left[\partialderivative{f^a(\bar{\theta},\tilde{\theta})}{\tilde{\theta}^c}\right]_{\tilde{\theta}=\theta_1}\right]_{\bar{\theta}=0}
	\theta_3^b	\left[h^{-1}\right]^c_{\hphantom{c}e}(\theta_1)\theta_2^e\\
	&=f^a(0,\theta_1)+\left[h^{-1}\right]^a_{\hphantom{a}b}(\theta_1)\theta_3^b+\left[h^{-1}\right]^a_{\hphantom{a}e}(\theta_1)\theta_2^e\\
	&+ \partialderivative{}{\theta_1^c}\left(\left[h^{-1}\right]^a_{\hphantom{a}b}(\theta_1)\right)\theta_3^b	\left[h^{-1}\right]^c_{\hphantom{c}e}(\theta_1)\theta_2^e\\
\end{align*}
No $\theta_3^2, \theta_2^2$ terms show up and \[\left[\partialderivative{f^a(0,\bar{\theta})}{\bar{\theta}^b}\right]_{\bar{\theta}=\theta_1}=\delta_b^a,\] see \eq{2.2.19}.
Equating coefficients of $\theta^b_3 \theta_2^c$, we get:
\begin{align*} 
	&\left[h^{-1}\right]^a_{\hphantom{a}e}(\theta_1)f^e_{\hphantom{e}bc}\\
	&=\partialderivative{}{\theta_1^d}\left(\left[h^{-1}\right]^a_{\hphantom{a}b}(\theta_1)\right)\left[h^{-1}\right]^d_{\hphantom{d}c}(\theta_1)\\
	\Leftrightarrow f^e_{\hphantom{e}bc}&=h^e_{\hphantom{e}a}(\theta_1)\partialderivative{}{\theta_1^d}\left(\left[h^{-1}\right]^a_{\hphantom{a}b}(\theta_1)\right)\left[h^{-1}\right]^d_{\hphantom{d}c}(\theta_1)\\
	&=\left(0-h^e_{\hphantom{e}a,d}\left[h^{-1}\right]^a_{\hphantom{a}b}\right)\left[h^{-1}\right]^d_{\hphantom{d}c}\\
	\Leftrightarrow h^e_{\hphantom{e}a,d}&=-f^e_{\hphantom{e}bc} h^b_{\hphantom{b}a}h^c_{\hphantom{c}d}
\end{align*}

\subsubsection{\eq{2.B.11} \page{98}}
\begin{align*} 
	h^a_{\hphantom{a}c,b}-h^a_{\hphantom{a}b,c}
	&\overset{\eqc{2.B.10}}{=}-f^a_{\hphantom{a}de} h^d_{\hphantom{d}c}h^e_{\hphantom{e}b}+f^a_{\hphantom{a}de} h^d_{\hphantom{d}b}h^e_{\hphantom{e}c}\\
	&= h^d_{\hphantom{d}c}h^e_{\hphantom{e}b}\left(-f^a_{\hphantom{a}de}+f^a_{\hphantom{a}ed}\right)\\
	&\overset{\eqc{2.2.23}}{=} h^d_{\hphantom{d}c}h^e_{\hphantom{e}b} \left(-C^a_{\hphantom{a}ed}\right)
\end{align*}

\subsubsection{\enquote{It follows that $U_\theta(1)$ is stationary under any infinitesimal variation of the path that leaves the endpoints $\Theta(0)=0$ and $\Theta(1)=\theta$ (and $U_\theta(0)=\identity$) fixed.} \page{98}} \label{sususec:2_B_p98_1}
We have 
\[U^{-1} \delta U -i U^{-1} t_a U h^a_{\hphantom{a}b}\delta \Theta^b = C = \text{const}\]
For $s=0$ this gives
\[0=\delta U(0) = i t_a U(0) h^a_{\hphantom{a}b}(0)  \underbrace{\delta\Theta^b (0)}_{=0} + U(0) C=\identity C\]
such that $C=0$.
For $s=1$ we then get \[\delta U(1) = i t_a U(1) h^a_{\hphantom{a}b}(1) \underbrace{\delta\Theta^b (1)}_{=0} + U(1) \underbrace{C}_{=0}=0.\]

\subsubsection{\enquote{$0=\partialderivative{}{\theta^b}\left\{U[\theta]^{-1}\tilde{U}[\theta]\right\}+i\phi_b(\theta) U[\theta]^{-1} \tilde{U}[\theta]$ where $\phi_b(\theta)=h^a_{\hphantom{a}b}\left[\partialderivative{\phi(\theta^\prime,\theta)}{\theta^{\prime a}}\right]_{\theta^\prime=0}$} \page{99}}\label{sususec:2_B_p99_1}
By taking the derivative of the previous equation w.r.t. $\theta^{\prime a}$, we obtain:
\begin{align*} 
	0&=U[\theta]^{-1} (-t_a+t_a)\tilde{U}[\theta]\\
	&=\left[\partialderivative{}{\theta^{\prime a}}U[\theta]^{-1}U[\theta^\prime]^{-1}\tilde{U}[\theta^\prime]\tilde{U}[\theta]\right]_{\theta^\prime=0}\\
	&=\left[\partialderivative{}{\theta^{\prime a}}U[f(\theta^\prime,\theta)]^{-1}\tilde{U}[f(\theta^\prime,\theta)]\exp(i\phi(\theta^\prime,\theta))\right]_{\theta^\prime=0}\\
	&=\left[\partialderivative{}{\bar{\theta}^b}U[\bar{\theta}]^{-1} \tilde{U}[\bar{\theta}]\right]_{\bar{\theta}=f(0,\theta)=\theta} \left[\partialderivative{f^b(\theta^\prime,\theta)}{\theta^{\prime a}}\right]_{\theta^\prime=0}\exp(i\phi(0,\theta))\\
	&+U[\theta]^{-1}\tilde{U}[\theta]i \left[\partialderivative{\phi(\theta^\prime,\theta)}{\theta^{\prime a}}\right]_{\theta^\prime=0}\exp(i\phi(0,\theta))\\
	&=\partialderivative{}{\theta^b}\left\{U[\theta]^{-1}\tilde{U}[\theta]\right\}\left[h^{-1}\right]^b_{\hphantom{b}a}(\theta)\\
	&+U[\theta]^{-1}\tilde{U}[\theta]i \left[\partialderivative{\phi(\theta^\prime,\theta)}{\theta^{\prime a}}\right]_{\theta^\prime=0}
\end{align*}
Multiplying by $h^a_{\hphantom{a}b}(\theta)$ yields finally
\[0=\partialderivative{}{\theta^b}\left\{U[\theta]^{-1}\tilde{U}[\theta]\right\}+i\phi_b(\theta) U[\theta]^{-1} \tilde{U}[\theta].\]

\subsubsection{\enquote{$0=\partialderivative{\phi_b(\theta)}{\theta^c}-\partialderivative{\phi_c(\theta)}{\theta^b}$} \page{99}}
Differentiating the result of \ref{sususec:2_B_p99_1} w.r.t $\theta ^c$ yields
\begin{align*} 
	0&=\partialderivative{^2}{\theta^c\partial\theta^b}\left\{U[\theta]^{-1}\tilde{U}[\theta]\right\}+i\partialderivative{\phi_b(\theta)}{\theta^c} U[\theta]^{-1} \tilde{U}[\theta]\\
	&+i\phi_b(\theta)\partialderivative{}{\theta^c}\left\{U[\theta]^{-1}\tilde{U}[\theta]\right\}\\
	&\overset{\ref{sususec:2_B_p99_1}}{=}\partialderivative{^2}{\theta^c\partial\theta^b}\left\{U[\theta]^{-1}\tilde{U}[\theta]\right\}+i\partialderivative{\phi_b(\theta)}{\theta^c} U[\theta]^{-1} \tilde{U}[\theta]\\
	&+\phi_b(\theta)\phi_c(\theta) U[\theta]^{-1} \tilde{U}[\theta].
\end{align*}
Antisymmetrizing then gives 
\[0=\partialderivative{\phi_b(\theta)}{\theta^c}-\partialderivative{\phi_c(\theta)}{\theta^b}.\]

\subsubsection{\enquote{Then $U^{-1}(f(\theta_2,\theta_1))U(\theta_2)U(\theta_1)$ can be a phase factor $\exp(i\phi(\theta_2,\theta_1))\neq 1$, but $\phi$ will be the same for all other loops into which this can be continuously deformed.} \page{99}}
This can be seen from the statement of \ref{sususec:2_B_p98_1} by setting $\theta =0$.



\subsection{Inversions and Degenerate Multiplets}\label{susec:2_C}

\subsubsection{\enquote{\dots the corresponding proportionality factor for $\timereversal^2$ can only be $\pm1$,\dots} \page{100}}
Suppose 
\[\timereversal^2=\varphi \identity\] 
then we have
\[\timereversal^3=\timereversal^2 \timereversal=\varphi \timereversal\]
but also
\[\timereversal^3=\timereversal \timereversal^2 =\timereversal\varphi=\varphi^\star \timereversal\]
such that
\[\varphi^\star=\varphi=\pm1.\]

\subsubsection{\enquote{\dots because \timereversal is anti unitary, \timereversalmatrix must be unitary} \page{101}}
Using basic orthonormality properties we see

\begin{align*}
	&\delta_{\sigma^\prime \sigma} \delta^{(3)}(\tvec{p}^\prime-\tvec{p})\delta_{ni}\\
	&\overset{\eqc{2.5.19}}{=}\innerproduct{\Psi_{\tvec{p}^\prime,\sigma^\prime,n}}{\Psi_{\tvec{p},\sigma,i}}^\star\\
	&=\innerproduct{\timereversal \Psi_{\tvec{p}^\prime,\sigma^\prime,n}}{\timereversal\Psi_{\tvec{p},\sigma,i}}\\
	&\overset{\eqc{2.C.1}}{=}\innerproduct{(-1)^{j^\prime-\sigma^\prime}\sum\limits_{m}\timereversalmatrix_{mn}\Psi_{-\tvec{p}^\prime,-\sigma^\prime,m}}{(-1)^{j-\sigma}\sum\limits_{l}\timereversalmatrix_{li}\Psi_{-\tvec{p},-\sigma,l}}\\
	&=(-1)^{j^\prime+j-\sigma^\prime-\sigma}\sum\limits_{m,l}\timereversalmatrix_{mn}^\star\timereversalmatrix_{li}\innerproduct{\Psi_{-\tvec{p}^\prime,-\sigma^\prime,m}}{\Psi_{-\tvec{p},-\sigma,l}}\\
	&\overset{\eqc{2.5.19}}{=}(-1)^{2(j-\sigma)}\sum\limits_{m}\timereversalmatrix_{mn}^\star\timereversalmatrix_{mi}\delta_{\sigma^\prime \sigma} \delta^{(3)}(\tvec{p}^\prime-\tvec{p})\\
	&=\delta_{\sigma^\prime \sigma} \delta^{(3)}(\tvec{p}^\prime-\tvec{p})
	\sum\limits_{m}\left(\timereversalmatrix^\dagger\right)_{nm}\timereversalmatrix_{mi}
\end{align*}
such that
\[\delta_{ni}=\sum\limits_{m}\left(\timereversalmatrix^\dagger\right)_{nm}\timereversalmatrix_{mi}.\]

\subsubsection{\eq{2.C.2} \page{101}}
\begin{align*}
	&\timereversal \Psi_{\tvec{p},\sigma,n}^\prime\\
	&=\sum\limits_{m}\timereversal\mathcal{U}_{mn}\Psi_{\tvec{p},\sigma,m}\\
	&=\sum\limits_{m}\mathcal{U}_{mn}^\star\timereversal\Psi_{\tvec{p},\sigma,m}\\
	&\overset{\eqc{2.C.1}}{=}\sum\limits_{m,k}\mathcal{U}_{mn}^\star(-1)^{j-\sigma}\timereversalmatrix_{km}\Psi_{-\tvec{p},-\sigma,k}\\
	&=\sum\limits_{m,k,l}\mathcal{U}_{mn}^\star(-1)^{j-\sigma}\timereversalmatrix_{km}\left(\mathcal{U}^{-1}\right)_{lk}\Psi_{-\tvec{p},-\sigma,l}^\prime\\
	&=(-1)^{j-\sigma}\sum\limits_{l}\left(\mathcal{U}^{-1}\timereversalmatrix \mathcal{U}^\star\right)_{ln}\Psi_{-\tvec{p},-\sigma,l}^\prime
\end{align*}

\subsubsection{\enquote{\dots \textbf{unitary} matrix $\timereversalmatrix \timereversalmatrix^\star$.} \page{101}}
\begin{align*}
	\left(\timereversalmatrix \timereversalmatrix^\star\right)^\dagger \timereversalmatrix \timereversalmatrix^\star&=\left(\timereversalmatrix^\dagger\right)^\star \timereversalmatrix^\dagger \timereversalmatrix \timereversalmatrix^\star\\
	&=\left(\timereversalmatrix^\dagger\right)^\star\timereversalmatrix^\star\\
	&=\left(\timereversalmatrix^\dagger\timereversalmatrix\right)^\star\\
	&=\identity^\star=\identity
\end{align*}

\subsubsection{\eq{2.C.4} \page{101}}
We have \[\timereversalmatrix\timereversalmatrix^\star=D\]
multiplying by $\timereversalmatrix^\top=\left(\timereversalmatrix^\dagger\right)^\star$ from the right we get
\[\timereversalmatrix=\timereversalmatrix\left(\timereversalmatrix\timereversalmatrix^\dagger\right)^\star=\timereversalmatrix\timereversalmatrix^\star\timereversalmatrix^\top=D\timereversalmatrix^\top.\]

\subsubsection{\enquote{\dots the diagonal element $\timereversalmatrix_{nn}$ vanishes unless $\exp(i\phi_n)=1$.Furthermore, if $\exp(i\phi_n)=1$ but $\exp(i\phi_m)\neq1$, then \eq{2.C4} tells us that $\timereversalmatrix_{nm}=\timereversalmatrix_{mn}=0$.} \page{101}}

From \eq{2.C.4} we have
\[\timereversalmatrix_{nm}=\exp(i\phi_n)\timereversalmatrix_{mn}.\]
From this we get
\begin{align*}
	\timereversalmatrix_{nn}&=\exp(i\phi_n)\timereversalmatrix_{nn}\\
	\Rightarrow 0&=\left(1-\exp(i\phi_n)\right)\timereversalmatrix_{nn}
\end{align*}
such that for $\exp(i\phi_n)\neq1$ we have \[\timereversalmatrix_{nn}=0.\]
Furthermore, if $\exp(i\phi_n)=1$ but $\exp(i\phi_m)\neq1$, then we have
\begin{align*}
	\timereversalmatrix_{mn}&=\exp(i\phi_m)\timereversalmatrix_{nm}\\
	&=\exp(i\phi_m)\exp(i\phi_n)\timereversalmatrix_{mn}\\
	&=\underbrace{\exp(i\phi_m)}_{\neq1}\timereversalmatrix_{mn}\\
	\Rightarrow \timereversalmatrix_{mn}&=\timereversalmatrix_{nm}=0
\end{align*}

\subsubsection{\enquote{\dots $\mathcal{A}$ is symmetric as well as unitary, \dots} \page{101}}

Unitarity follows directly from the unitarity of \timereversalmatrix and symmetry can be seen from \[\timereversalmatrix_{nm}=\exp(i\phi_n)\timereversalmatrix_{mn}\] because $\mathcal{A}$ only contains rows and columns for which $\exp(i\phi_n)=1$.


\subsubsection{\enquote{Because $\mathcal{A}$ is symmetric, it can be expressed as the exponential of a symmetric anti-Hermitian matrix, so it can be diagonalized by a transformation \eq{2.C.2} acting on $\mathcal{A}$, with the corresponding submatrix of $\mathcal{U}$ real and hence orthogonal.} \page{101}}

We know that $\mathcal{A}$ is unitary and symmetric, i.e.
\[\mathcal{A}^\dagger=\mathcal{A}^{-1}\qquad \mathcal{A}^\top=\mathcal{A}.\]
Suppose $\mathcal{A}$ can be written as \[\mathcal{A}=\exp(a)\] where $a$ is a symmetric anti-Hermitian matrix, i.e.
\[a^\dagger=-a\qquad a^\top=a.\]
We can easily check that this satisfies all properties imposed on $\mathcal{A}$:
\begin{align*}
	\mathcal{A}^\dagger \mathcal{A}&=\exp(a^\dagger)\exp(a)\\
	&=\exp(-a)\exp(a)\\
	&=\exp(-a+a)=\identity\\
	\mathcal{A}^\top&=\exp(a^\top)=\exp(a)=\mathcal{A}
\end{align*}
Any symmetric anti-Hermitian matrix $a$ can be written in terms of a symmetric Hermitian matrix $h$ as \[a=ih,\]
such that \[\mathcal{A}=\exp(a)=\exp(i h).\]
Observe that $h$ is real, since
\[h^\star=\left(h^\top\right)^\dagger=h^\dagger=h.\]
Since $h$ is real and by definition symmetric, it can be diagonalized by an orthogonal matrix:
\[h=O^{-1}D O\]
Inserting this into the expression for $\mathcal{A}$ we see that $O$ also diagonalizes  $\mathcal{A}$:
\begin{align*}
	\mathcal{A}&=\exp(i h)=\exp(i O^{-1}D O)\\
	&=\exp( O^{-1}iD O)=O^{-1}\exp(iD)O
\end{align*}
Since $O$ is orthogonal and in particular real it can be set as a submatrix of $\mathcal{U}$ in the transformation \eq{2.C.2}.


\subsubsection{\eq{2.C.7} \page{102}}
For components of \timereversalmatrix in the same block $\mathcal{B}_i$ we have
\[\exp(i\phi_m)=\exp(-i\phi_n).\]
Such that pulling a factor of $\exp(-i\dfrac{\phi_n}{2})$ out of $\timereversalmatrix_{mn}$ we can write
\begin{align*}
	\timereversalmatrix_{mn}&=\exp(-i\dfrac{\phi_n}{2})z\\
	\Rightarrow \timereversalmatrix_{nm}&=\exp(i\phi_n)\exp(-i\dfrac{\phi_n}{2})z\\
	&=\exp(i\dfrac{\phi_n}{2})z
\end{align*}
with $z$ some complex number specific to the combination of indices $m,n$.


\subsubsection{\enquote{\dots $\mathcal{C}_i\mathcal{C}_i^\dagger=\mathcal{C}_i^\dagger \mathcal{C}_i=\identity$, and hence $\mathcal{C}_i$ is square and unitary} \page{102}}

The Unitarity of \timereversalmatrix implies the Unitarity of $\mathcal{B}$ which in turn implies the Unitarity of each $\mathcal{B}_i$. This imposes the following condition:
\begin{widetext}
	\begin{align*}
		\mathcal{B}_i^\dagger \mathcal{B}_i&=
		\left(\begin{array}{cc}
			0	&  \exp(i\dfrac{\phi_n}{2})\mathcal{C}_i^\star\\
			\exp(-i\dfrac{\phi_n}{2})\mathcal{C}_i^\dagger	& 0
		\end{array}\right)
		\left(\begin{array}{cc}
			0	&  \exp(i\dfrac{\phi_n}{2})\mathcal{C}_i\\
			\exp(-i\dfrac{\phi_n}{2})\mathcal{C}_i^\top	& 0
		\end{array}\right)\\
	&=\left(\begin{array}{cc}
		\mathcal{C}_i^\star \mathcal{C}_i^\top	&  0\\
		0	& \mathcal{C}_i^\dagger \mathcal{C}_i
	\end{array}\right)
=\left(\begin{array}{cc}
\left(	\mathcal{C}_i \mathcal{C}_i^\dagger \right)^\top	&  0\\
 	0	& \mathcal{C}_i^\dagger \mathcal{C}_i
\end{array}\right)
\overset{!}{=}\identity\\
\Rightarrow \mathcal{C}_i^\dagger \mathcal{C}_i&=\identity=\identity^\top=\mathcal{C}_i\mathcal{C}_i^\dagger
	\end{align*}
\end{widetext}


\subsubsection{\enquote{\dots $\exp(\pm i \dfrac{\phi}{2}) c^\star_\pm=\abs{\lambda}^2 c_\pm^\star \exp(\mp i \dfrac{\phi}{2})$, which is impossible unless either $c_+=c_-=0$ or $\exp(i\phi)$ is unity,\dots} \page{103}}

We have
\begin{align*}
 	\exp(i\dfrac{\phi}{2})c_+^\star = \lambda c_- &\Rightarrow c_+=\lambda^\star c_-^\star \exp(i\dfrac{\phi}{2})\\
 	\exp(-i\dfrac{\phi}{2})c_-^\star = \lambda c_+ &\Rightarrow c_-=\lambda^\star c_+^\star \exp(-i\dfrac{\phi}{2})
\end{align*}
such that
\begin{align*}
	\exp(\pm i \dfrac{\phi}{2}) c^\star_\pm&=\lambda c_\mp\\
	&=\abs{\lambda}^2 c^\star_\pm \exp(\mp i \dfrac{\phi}{2})
\end{align*}
which is equivalent to 
\[\exp(\pm i {\phi}) c^\star_\pm=\abs{\lambda}^2 c^\star_\pm.\]
This is only possible if either $c_+=c_-=0$ or $\exp(i\phi)$ is unity.

\subsubsection{\eq{2.C.16} \page{104}}
From \eq{2.C.8} we have
\begin{align*}
	\Psi_{\tvec{p},\sigma,\pm}&=\timereversal^{-1} \timereversal \Psi_{\tvec{p},\sigma,\pm}\\
	=\exp(\mp i \dfrac{\phi}{2})&(-1)^{j-\sigma} \timereversal^{-1}\Psi_{-\tvec{p},-\sigma,\mp}\\
	\Rightarrow\timereversal^{-1}\Psi_{-\tvec{p},-\sigma,\mp}&=\exp(\pm i \dfrac{\phi}{2})(-1)^{-j+\sigma}\Psi_{\tvec{p},\sigma,\pm}\\
	\Rightarrow\timereversal^{-1}\Psi_{\tvec{p},\sigma,\pm}&=\exp(\mp i \dfrac{\phi}{2})(-1)^{-j-\sigma}\Psi_{-\tvec{p},-\sigma,\mp}
\end{align*}
with this we get:
\begin{align*}
	\chargeconjugate \parity \Psi_{\tvec{p},\sigma,\pm} &=  \left(\chargeconjugate \parity \timereversal\right) \timereversal^{-1}\Psi_{\tvec{p},\sigma,\pm}\\
	&=\left(\chargeconjugate \parity \timereversal\right)\exp(\mp i \dfrac{\phi}{2})(-1)^{-j-\sigma}\Psi_{-\tvec{p},-\sigma,\mp}\\
	&=\exp(\pm i \dfrac{\phi}{2})(-1)^{-j-\sigma}\left(\chargeconjugate \parity \timereversal\right)\Psi_{-\tvec{p},-\sigma,\mp}\\
	&=\exp(\pm i \dfrac{\phi}{2})(-1)^{-j-\sigma}(-1)^{j+\sigma}\Psi_{-\tvec{p},\sigma,\mp^C}\\
	&=\exp(\pm i \dfrac{\phi}{2})\Psi_{-\tvec{p},\sigma,\mp^C}
\end{align*}

\subsection{Problems}\label{susec:2_problems}
\subsubsection{Problem}
